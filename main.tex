%%%
% Universidade Estadual do Oeste do Paraná - Unioeste
% Centro de Ciências Exatas e Tecnológicas
% Curso de Bacharelado em Ciência da Computação
% Conteúdo: Arquivo principal para a compilação do TCC
% Obs.: Para maiores personalizações, veja o arquivo "unioeste.sty"
%%%
\documentclass[12pt, oneside, a4paper, brazil]{book}

% ---
% Tratamento de acentos / charset
% ---
\usepackage[T1]{fontenc}
% Em caso de problemas na compilação, tente alternar os packages abaixo:
% \usepackage[latin1]{inputenc}
\usepackage[utf8]{inputenc}
% ---

\usepackage{unioeste}

% ---
% document
% ---
\begin{document}

% numeracao em romanos para paginas iniciais (capa, agradecimentos, etc)
\pagestyle{empty}
\pagenumbering{roman}

% paginas iniciais
\begin{wrapfigure}[4]{i}{0cm}
	\centering
	\includegraphics[width = 3.13cm, height = 2.19cm]{Figuras/Simbolo.jpg}
\end{wrapfigure}
\fontsize{13}{13}
\noindent
\textbf{Unioeste - Universidade Estadual do Oeste do Paran\'{a}}\\
\fontsize{11}{11}
CENTRO DE CI\^{E}NCIAS EXATAS E TECNOL\'{O}GICAS\\
Colegiado de Ci\^{e}ncia da Computa\c{c}\~{a}o\\
\textbf{\textit{Curso de Bacharelado em Ci\^{e}ncia da Computa\c{c}\~{a}o}}
\vspace{9cm}
\begin{center}
\fontsize{13}{13}
\textbf{E4J: Editor i* para JGOOSE}\\
% \textbf{Desenvolvendo e Integrando um Editor i* ao JGOOSE}\\
\vspace{0.3cm}
\textit{Leonardo Pereira Merlin}\\
\vspace{9cm}
\textbf{CASCAVEL}\\
\textbf{2013}
\end{center}

% Universidade Estadual do Oeste do Paran\'{a} - UNIOESTE
% Centro de Ci\^{e}ncias Exatas e Tecnol\'{o}gicas
% Curso de Bacharelado em Inform\'{a}tica
% Arquivo: Rosto.Tex
% Conte\'{u}do: Folha de Rosto para o Arquivo do TCC.

%%%%%%%%%%%%%%%%%%%%%%%%%%%%%%%%%%%%%%%%%%%%%%%%%%%%%%%%%%%%%%%%%%%%%
%%%
%%% Folha de Rosto
%%%
%%%%%%%%%%%%%%%%%%%%%%%%%%%%%%%%%%%%%%%%%%%%%%%%%%%%%%%%%%%%%%%%%%%%%

\fontsize{12}{12}
\begin{center}
\textbf{Leonardo Pereira Merlin}\\
\vspace{8cm}
\fontsize{14}{14}
\textbf{E4J: Editor i* para JGOOSE}\\
%\textbf{Desenvolvendo e Integrando um Editor i* ao JGOOSE}\\
\vspace{2cm}
\end{center}
\fontsize{12}{12}

\begin{flushright}
\begin{minipage}[10cm] {8.5cm}
Monografia apresentada como requisito parcial para obten\c{c}\~{a}o do grau de Bacharel em Ci\^{e}ncia da Computa\c{c}\~{a}o, do Centro de Ci\^{e}ncias Exatas e Tecnol\'{o}gicas da Universidade Estadual do Oeste do Paran\'{a} - Campus de Cascavel

\vspace{1.5cm}
\noindent
Orientador: Prof. Victor Francisco Araya Santander
\end{minipage}
\end{flushright}

\vspace{5.5cm}
\begin{center}
CASCAVEL\\
2013
\end{center}

%%%%%%%%%%%%%%%%%%%%%%%%%%%%%%%%%%%%%%%%%%%%%%%%%%%%%%%%%%%%%%%%%%%%%
%%% folha de aprova\c{c}\~{a}o
%%%%%%%%%%%%%%%%%%%%%%%%%%%%%%%%%%%%%%%%%%%%%%%%%%%%%%%%%%%%%%%%%%%%%

\begin{center}
\fontsize{12}{12}
\textbf{Leonardo Pereira Merlin}\\
\vspace{3cm}
\fontsize{14}{14}
\textbf{Desenvolvendo e Integrando um Editor i* ao JGOOSE}\\
\vspace{3cm}
\fontsize{10}{10}
Monografia apresentada como requisito parcial para obten\c{c}\~{a}o do T\'{\i}tulo de Bacharel em Ci\^{e}ncia da Computa\c{c}\~{a}o, pela Universidade Estadual do Oeste do Paran\'{a}, Campus de Cascavel, aprovada pela Comiss\~{a}o formada pelos professores:\\
\vspace{2cm}
\begin{flushright}
\begin{minipage}[10cm] {8.5cm}
\begin{center}
\rule{6cm}{0.01mm}\\
Prof. Victor Francisco Araya Santander\\
Colegiado de Ci\^{e}ncia da Computa\c{c}\~{a}o, UNIOESTE\\
\vspace{1cm}
\rule{6cm}{0.01mm}\\
Prof. Ivonei Freitas da Silva\\
Colegiado de Ci\^{e}ncia da Computa\c{c}\~{a}o, UNIOESTE\\
\vspace{1cm}
\rule{6cm}{0.01mm}\\
Prof. Elder Elisandro Schemberger\\
Colegiado de Ci\^{e}ncia da Computa\c{c}\~{a}o, UNIOESTE\\
\end{center}
\end{minipage}
\end{flushright}
\vspace{3.5cm}
Cascavel, 31 de maio de 2013
\end{center} 

\oneandhalfspacing

%Itens opcionais
%%%%%%%%%%%%%%%%%%%%%%%%%%%%%%%%%%%%%%%%%%%%%%%%%%%%%%%%%%%%%%%%%%%%%
%%%
%%% Dedicat\'{o}ria
%%%
%%%%%%%%%%%%%%%%%%%%%%%%%%%%%%%%%%%%%%%%%%%%%%%%%%%%%%%%%%%%%%%%%%%%%

\begin{center}
\fontsize{14}{14}
%\textbf{DEDICAT\'{O}RIA}
\end{center}

\vspace{10cm}

\begin{flushright}
  \begin{minipage}[10cm] {8.5cm}
  \emph {Oscar Jos\'{e} Merlin J\'{u}nior, dedico este trabalho a voc\^{e}, meu irm\~{a}o e minha refer\^{e}ncia. }
  \end{minipage}
\end{flushright}

\begin{center}
\fontsize{14}{14}
%\textbf{EP\'{I}GRAFE}
\end{center}

\vspace{10cm}

\begin{flushright}
\begin{minipage}[10cm] {8.5cm}

  \emph {"Se voc\^{e} pensa que pode ou se pensa que n\~{a}o pode, de qualquer forma voc\^{e} est\'{a} certo."
	\\Henry Ford}

\end{minipage}
\end{flushright}

%%%%%%%%%%%%%%%%%%%%%%%%%%%%%%%%%%%%%%%%%%%%%%%%%%%%%%%%%%%%%%%%%%%%%
%%%
%%% Folha de Agradecimentos
%%%
%%%%%%%%%%%%%%%%%%%%%%%%%%%%%%%%%%%%%%%%%%%%%%%%%%%%%%%%%%%%%%%%%%%%%

\begin{center}
\fontsize{14}{14}
\textbf{AGRADECIMENTOS}
\end{center}
\vspace{2cm}

Em primeiro lugar, eu gostaria de agradecer ao meu orientador, professor Victor Francisco Araya Santander, pelos conselhos, al\'{e}m do apoio, disponibilidade e interesse no meu trabalho.
Apresentou-me oportunidades que eu jamais acreditava que teria novamente.
E, al\'{e}m de acreditar nos resultados, sempre me inspirou coragem e \^{a}nimo para cumprir meus objetivos.

Tamb\'{e}m gostaria de agradecer aos outros membros da banca examinadora, os professores Ivonei Freitas da Silva e Elder Elisandro Schemberger. Obrigado por analisar e avaliar cuidadosamente o meu trabalho. Durante as apresenta\c{c}\~{o}es, contribuiram significativamente com questionamentos e sugest\~{o}es, de fato, relevantes. E n\~{a}o distante desses, meus sinceros agradecimentos a todos os acad\^{e}micos integrantes do grupo de estudo do Laborat\'{o}rio de Engenharia de Software (LES). Dentre esses, um agradecimento especial aos companheiros Diego Peliser, Leonardo Zanotto Baggio e Rodrigo Trage. Obrigado pelas apresenta\c{c}\~{o}es e discuss\~{o}es sobre temas de destaque na \'{a}rea de Engenharia de Software. Espero poder retribuir e contribuir para com o grupo.

Quanto aos colegas de moradia, Lucas In\'{a}cio, Bruno Belorte, Marcos Schmitt, Nicolas Zaro e Julian Ruiz Diaz, obrigado pela toler\^{a}ncia e bons momentos.

N\~{a}o posso deixar de agradecer os amigos Fernando Dal Bello, Amadeu Paix\~{a}o e Felipe Carminati pelas experi\^{e}ncias profissionais, oportunidades de um trabalho em equipe bem realizado e um amadurecimento pessoal memor\'{a}vel (nada que uma boa trilha sonora n\~{a}o resolva). Em especial, agrade\c{c}o ao Carlos Henrique de Fran\c{c}a, que me ajudou a esclarecer e agu\c{c}ar minhas pesquisas, al\'{e}m das cr\'{\i}ticas em numerosas vers\~{o}es de documentos t\'{e}cnicos.

% Os grandes amigos "imagin\'{a}rios", Victor Hudo e Roberto Pacheco Leal da Silva, que sempre me incentivaram das mais variadas maneiras. Esse Brasil \'{e} pequeno de mais para nos separar das \'{o}timas aventuras.

A minha fam\'{\i}lia, pelos conselhos (n\~{a}o ignorados), pelo aux\'{\i}lio e suporte de diversas maneiras. E, Lah (Larissa Torquato de Oliveira e fam\'{\i}lia), este lugar tamb\'{e}m \'{e} de voc\^{e}s.

A minha companheira, Jamile Merlin, por participar de mais esta fase da minha vida.

\`{A} todos, meu sincero agradecimento!

% Agradecer Portal da CAPES? pelos ARTIGOS - GOVERNO?


% Volta a numeração arábica
\pagestyle{plain}
%%%%%%%%%%%%%%%%%%%%%%%%%%%%%%%%%%%%%%%%%%%%%%%%%%%%%%%%%%%%%%%%%%%%%
%%%
%%% Lista de Figuras
%%% Lista de Tabelas
%%% Lista de S\'{\i}mbolos
%%% Sum\'{a}rio
%%%
%%%%%%%%%%%%%%%%%%%%%%%%%%%%%%%%%%%%%%%%%%%%%%%%%%%%%%%%%%%%%%%%%%%%%

\pagebreak
\addcontentsline{toc}{chapter}{Lista de Figuras}
\listoffigures

%\pagebreak
%\addcontentsline{toc}{chapter}{Lista de Tabelas}
%\listoftables

\pagebreak
\addcontentsline{toc}{chapter}{Lista de Abreviaturas e Siglas}
\chapter*{Lista de Abreviaturas e Siglas}
\begin{tabular}{ll}
	API			& \textit{Application Programming Interface}\\
	DFD			& Diagrama de Fluxo de Dados\\
    EMF         & \textit{Eclipse Modelling Framework}\\
    ES          & Enngenharia de Software\\
    GOOSE       & \textit{Goal Into Object Oriented Standard Extension }\\
    GUI         & \textit{Graphical User Interface}\\
	IDE			& \textit{Integrated Development Environment}\\
	ITU			& \textit{International Telecommunication Union}\\
	JGOSOE		& \textit{Java Goal Into Object Oriented Standard Extension }\\
    OME         & \textit{Organization Modelling Environment }\\
    ORM         & \textit{Object/Relational Mapping}\\
    POM         & \textit{Project Object Model}\\
	POO			& Paradigma Orientado a Objetos\\
	% SGBD		& Sistema Gerenciador de Banco de Dados\\
    % SO          & Sistemas Operacionais\\
    SD          & Modelo de Dependências Estratégicas\\
    SR          & Modelo de Razões Estratégicas\\
	UML			& \textit{Unified Modeling Language}\\
	XMI			& \textit{Xml Metadata Interchange}\\
	XML			& \textit{eXtensible Markup Language}\\
	% UNIOESTE 	& Universidade Estadual do Oeste do Paraná\\
\end{tabular}

%\pagebreak
%\addcontentsline{toc}{chapter}{Lista de S\'{\i}mbolos}
%\chapter*{Lista de S\'{\i}mbolos}
%\begin{tabular}{ll}

%%
%% Humanos
%%

% Compartimentos
%	$Sh$ & Total de humanos no estado suscet\'{\i}vel\\
%	$Eh$ & Total de humanos no estado exposto\\
%	$Ih$ & Total de humanos no estado infectante\\
%	$Rh$ & Total de humanos no estado recuperado e renovado\\
		

%\end{tabular}
\pagebreak
\addcontentsline{toc}{chapter}{Sum\'{a}rio}
\tableofcontents
 % Listas(Figuras,Tabelas,Smbols) e Sumrio
\chapter*{Resumo}
\addcontentsline{toc}{chapter}{Resumo}
\noindent

Em discuss\~{o}es  pertinentes  \`{a}  engenharia de requisitos, tanto em \^{a}mbito
acad\^{e}mico quanto industrial, ressalta-se que um dos principais desafios da \'{a}rea
consiste na integra\c{c}\~{a}o de modelos organizacionais \`{a}s demais  etapas do processo
de engenharia de requisitos.

Alguns trabalhos relacionados j\'{a} apresentaram t\'{e}cnicas para a realiza\c{c}\~{a}o dessa integra\c{c}\~{a}o. Nesse sentido, em 2006 foi
desenvolvida uma solu\c{c}\~{a}o computacional denominada JGOOSE (do ingl\^{e}s  Java Goal Into Object  Oriented Standard Extension),
uma ferramenta de aux\'{\i}lio a engenharia de requisitos que proporciona a automatiza\c{c}\~{a}o do 
mapeamento de modelos e diagramas do  framework  i* para casos de uso UML.

Apesar das atualiza\c{c}\~{o}es e melhorias realizada por outros pesquisadores e membros
do grupo do Laborat\'{o}rio de Engenharia de Software (LES) da Universidade Estadual do Oeste do Paran\'{a} (UNIOESTE - Campus Cascavel/PR),
a ferramenta ainda possui uma depend\^{e}ncia cr\'{\i}tica quanto a elabora\c{c}\~{a}o dos dados de entrada.
Ou seja, nas condi\c{c}\~{o}es atuais \'{e} necess\'{a}rio instalar alguma ferramenta espec\'{\i}fica para produzir o arquivo no formato TELOS.

Desta forma, este trabalho consiste no projeto e desenvolvimento de um editor gr\'{a}fico de modelos organizacionais i* integrado  \`{a} ferramenta JGOOSE, visando melhorar
suas funcionalidades e diminuir a necessidade de outros softwares para este fim.
Uma caracter\'{\i}stica importante desse novo editor \'{e} o suporte \`{a} especifica\c{c}\~{a}o
iStarML - um formato de arquivo baseado em XML para representa\c{c}\~{a}o de modelos i*.


\vspace{1cm}

\noindent
\textbf{Palavras-chave: } JGOOSE, iStarML, i* Framework, Modelagem Organizacional, Engenharia de Requisitos.

%=======================================================================
% Definir um estilo para páginas completas (cabealho + rodapé)
\pagestyle{plain}

% Marca de captulo do tipo "2. Blblbl..." no cabealho
\renewcommand{\chaptermark}[1]{\markboth{\thechapter.\ {#1}}{}}

% Para aumentar o tamanho da caixa para o cabealho
\addtolength{\headheight}{\baselineskip}

% Definindo o contedo do cabealho...
\fancyhf{}
\fancyhead[LO,LE]{\nouppercase{\textsf{\leftmark}}}
\fancyhead[RO,RE]{\thepage}


\fancypagestyle{plain}{
	% Para garantir que a primeira página de cada capítulo
	% não contenha nada (paginação, cabeçalho, rodapé)
	\fancyhf{} % clear all six fields
	\renewcommand{\headrulewidth}{0pt}
}

%=======================================================
% Capitulos sao numerados arabicamente
\pagenumbering{arabic}

\oneandhalfspacing
% Inclusão dos capítulos
% Capítulo 1
\chapter{Introdução}
    \label{cap:introducao}
        % intro
            Este primeiro capítulo tem como objetivo a apresentação geral do trabalho.
            É realizada a contextualização e delimitação da pesquisa ao escopo da Engenharia de Software,
            bem como
            são destacados os principais objetivos da proposta.
        % topicos
            % contexto <#ID_883627655>
                Apresenta-se inicialmente,
                na seção \ref{cap:introducao:sec:contexto},
                o contexto sobre as ferramentas de modelagem organizacional e suas contribuições na área da Engenharia de Requisitos, destacando a influência dessas ferramentas no desenvolvimento de produtos de qualidade.
            % motivacao <#ID_828369918>
                Na seção \ref{cap:introducao:sec:motivacao}, são apresentadas as principais influências e motivações para a realização do trabalho.
            % proposta <#ID_1542036198>
                Em seguida,
                na seção \ref{cap:introducao:sec:proposta},
                é apresentada a proposta sob uma visão geral e os objetivos norteadores da pesquisa.
            % contribuicoes <#ID_1701692394>
                Na seção \ref{cap:introducao:sec:contribuicoes},
                 descreve-se as contribuições esperadas após a finalização deste trabalho.
            % organizacao <#ID_1115413878>
                Por fim,
                 na seção \ref{cap:introducao:sec:organizacao},
                é apresentada a estrutura geral e a organização do restante desta monografia.
    \section{Contexto}
        \label{cap:introducao:sec:contexto}
        % Contexto - Ferramentas
            Muitas são as opções de ferramentas e técnincas que visam auxiliar engenheiros de requisitos no processo de construção de modelos organizacionais i* \cite{site2013iwiki} \cite{grau2006comparative}.
            Podendo ser classificadas como ferramentas CASE (\emph{Computer-Aided Software Engineer})
                \footnote{Ferramentas CASE, é toda e qualquer ferramenta baseada em computador que auxilie nas atividades de desenvolvimento de software.}
             \cite{case1985computer}, essas ferramentas têm como objetivo o aumento da produtividade e a melhoria da qualidade final dos softwares,
                 através da automatização e gerenciamento de várias fases da Engenharia de Software.
        
        % Contexto - E.R.
            A área de Engenharia de Requisitos (ER), subárea da Engenharia de Software, é responsável por diversas atividades que abrangem os processos de análise, elicitação, especificação, avaliação, ajuste, documentação e evolução dos requisitos de um sistema computacional.
            É uma das áreas mais críticas para o sucesso e qualidade de um projeto de software \cite{sommerville1998requirements}.
            Pesquisas pertinentes à ER, tanto em âmbito acadêmico quanto industrial,
            apontam a falta de um entendimento adequado da organização por parte dos responsáveis pela elaboração do documento de requisitos
            como sendo uma das principais falhas no processo de especificação dos requisitos
                \cite{van2000requirements}.

        % Contexto - solução: modelagem organizacional
            % intro
                Para tentar diminuir os problemas relacionados as fases iniciais do projeto,
                pesquisas recentes mostram que a comunidade tem buscado estabelecer e utilizar padrões de técnicas, métodos e ferramentas para tratar especificamente
                da fase inicial de desenvolvimento de software
                    % ref
                        % A Literature Survey on International Standards for Systems Requirements Engineering
                        \cite{schneider2013literature}.
                Pensando nisso,
                têm-se investido esforços no processo de modelagem organizacional.
            % conceito
                Este tipo de modelagem
                visa prover recursos que permitam modelar
                as intenções, relacionamentos e motivações
                entre membros de uma organização
                % refs
                    % A Conceptual Basis for Organizational Modelling
                    % mason1997conceptual <workspace:/../../../../E:/unioeste/BCC/TCC-2013/referencias/mason1997conceptual.pdf>
                    \cite{mason1997conceptual}.
            % modelagem i*
                Dentre as técnicas de modelagem organizacional,
                destaca-se a i*,
                proposto por
                    \cite{yu1993modeling},
                uma técnica que utiliza a orientação a agentes
                    \cite{yu2001agent}
                %
                    com enfoque
                    tanto nos desejos e intenções desses agentes, quanto suas dependências
                    %refs
                        \cite{site2013iwiki}
                        \cite{yu1997towards}.
                Mais detalhes sobre esta técnica e suas variações serão discutidos no capítulo
                \ref{cap:framework}.
        % Contexto - Mapeamento Modelagem / UseCase UML

            Pensando em auxiliar no processo de desenvolvimento de software,
            alguns trabalhos foram propostos com o intuito de realizar o mapeamento
            de modelos do \emph{framework} i*
            para diagramas da UML (do inglês \emph{Unified   Modeling   Language}).
            Dentre esses trabalhos, destaca-se o trabalho de Santander \cite{santander2002integrando},
            que propõe a derivação em casos de uso UML a partir de modelos do \emph{framework} i*.
        % Contexto - JGOOSE
            % intro
                Para apoiar esse processo de derivação,
                foi desenvolvida a ferramenta JGOOSE (Java Goal into Object Oriented Standard Extension) \cite{vicente2006},
                uma ferramenta que mapeia de forma automática os diagramas i* para casos de uso UML.
                Essa ferramenta tem como base as diretrizes e passos propostos por Santander \cite{santander2002integrando},


                Inicialmente apresentada como GOOSE (\emph{Goal into Object Oriented Standard Extension}) em
                    \cite{pedroza2004ferramentas}
                    e
                    \cite{brischke2005desenvolvimento},
                em seguida, melhorada e apresentada como JGOOSE por
                    \cite{vicente2009},
                passou também por melhorias com
                    \cite{brischke2012melhorando}
                e, atualmente,
                está sendo aprimorada por
                    \cite{peliser2013aprimorando}.

                Ou seja, dado como entrada os modelos i*, no formato de arquivo TELOS
                    \cite{mylopoulos1990telos}
                    \cite{koubarakis1989telos},
                a ferramenta consegue gerar conforme o template proposto em
                    %ref
                    \cite{cockburn2001writing}
                os casos de uso UML com um bom nível de detalhamento.
            % problema 1 - dependência

                Porém, a ferramenta JGOOSE ainda não possui funcionalidades para a produção dos arquivos de entrada da ferramenta.
                Ainda existe a dependência da ferramenta OME ou, mais especificamente, ao formato de arquivo TELOS.
            % solução 1 - editor integrado
                Nesse contexto, percebe-se a necessidade de se desenvolver um editor de modelos i* integrado à ferramenta JGOOSE,
            % solução 2 - iStarML
                bem como implementar o suporte à especificação do formato de arquivo iStarML
                    % referenciar a proposta inicial do istarml
                    \cite{cares2007istarml},
                uma formato em XML
                para representação de modelos i*
                com o propósito de servir como um intercâmbio entre os outros meta-modelos existentes
                    \cite{colomer2011model}.
    \section{Motivação}
        \label{cap:introducao:sec:motivacao}
        % backlink - resumo <#ID_1432075869>
        % O que me motivou?
            % Motivo, Causa, Razão ou Circunstância
            % É uma área em crescimento e destaque. Muito esforço já se investiu em ferramentas CASE.
            % PQ é valido o esforço para aprimorar a ferramenta?
        % Motivacional 0 - Resolver algum dos Problemas mencionados no contexto
            % JGOOSE não possui um editor i*
            % Dependência de outras ferramentas p/ ler Telos
            % iStarML
        % Motivacional 1 - Ajudar a comunidade
            A área de Engenharia de Requisitos está em crescimento e destaque
            por impactar de forma tão significativa nos resultados finais de um projeto de software.
            Desta forma,
            é valido o investimento de esforços para a melhoria de métodos, técnicas ou ferramentas que auxiliem os profissionais da área a aprimorar seu trabalho de forma eficiente.
            % framework i*
                % padrão internacional
                    O i* é a base da GRL (\emph{Goal-oriented Requirements Language} ou Linguagem de Requisitos Orientada a Objetivos),
                    que junto à UCM
                        \footnote{UCM - \emph{Use Case Maps} , em português: ``Mapas de Caso de Uso''. Uma técnica de engenharia de software baseada em cenários para descrever relacionamentos entre um ou mais casos de uso.}
                    constituíram a URN
                        \footnote{URN - \emph{User Requirements Notation}, em portugês: ``Notação Requisitos de Usuário''. Notação destinada a elicitação, análise, especificação e validação de requisitos.},
                    que passou a ser adotada como um padrão internacional,
                    em novembro de 2008,
                    pela ITU (\emph{International Telecommunication Union})
                    % Z.151 : User Requirements Notation (URN) - Language definition
                    % refs
                        \cite{amyot2003introduction}
                        \cite{itu2003urn}. % http://www.itu.int/rec/T-REC-Z.151/en
                % motivação
                    Além de se tratar de um padrão internacional,
                    é a técnica de modelagem já justificada pela JGOOSE
                    e seus usuários já estão familiarizados com os conceitos do \emph{framework}.
        % Motivacional 2 - Ajudar os alunos

            Além disso,
            a ferramenta de que trata este trabalho
            é frequentemente usada por acadêmicos do curso de Ciência da Computação da Universidade Estadual do Oeste do Paraná - Campus Cascavel.
            Todos os anos, alunos se debatem com problemas apresentados por outros softwares de modelagem i*.
            Os questionamentos mais comuns estão relacionados à usabilidade e integridade das ferramentas.
            Isso acarreta em oportunidades para novas soluções computacionais se apresentarem.
        % Motivacional 3 - Estudar sobre a área

            Outro fator, não menos importante,
            é o gosto pessoal pela área de projeto e desenvolvimento de software.
            Isto ajudou na tomada de decisão quanto ao foco da pesquisa, bem como resultou na implementação de uma API para o iStarML (Veja o apêndice \ref{apendice:istarml}).
    \section{Proposta}
        \label{cap:introducao:sec:proposta}
        % backlink - resumo <#ID_1921781916>
        % intro JGOOSE
            A ferramenta JGOOSE,
            no escopo do seu propósito,
            já atende as principais necessidades do engenheiro de requisitos.
            Porém, ainda existe a dependência da ferramenta mencionada (OME)
            para elaborar os modelos organizacionais e exportá-los em arquivo TELOS.
        % estender JGOOSE (Obj. Geral)

            Dessa forma, o objetivo geral deste trabalho consiste em
            aumentar os recursos e funcionalidades da ferramenta JGOOSE através do
            desenvolvimento de uma nova interface para edição de modelos do framework i* integrada à JGOOSE.
            Ou seja, prover aos usuários da ferramenta JGOOSE uma interface gráfica rica em recursos que facilitem o trabalho de modelagem organizacional, visando diminuir a necessidade de usar outros softwares para esse fim.
        %Objetivos Específicos
            
             Como objetivos específicos, têm-se:
            \begin{itemize}
            % estudar literatura da área
                \item Estudar sobre as ferramentas de modelagem organizacional i* disponíveis à comunidade.
            % estudar sobre o framework i* e suas variações
                \item Estudar sobre o framework i*, bem como suas variações.
            % estudar a iStarML
                \item Estudar o formato de arquivo iStarML e incorporá-lo à ferramenta como o formato de arquivo padrão.
            % estudar a JGOOSE e sua evolução/histórico
                \item Realizar um estudo sobre a evolução histórica da ferramenta JGOOSE e analisar sua arquitetura na versão 2013.
            % criar e apresentar exemplos de uso da ferramenta
                \item Criar e apresentar exemplos de uso da ferramenta proposta, a fim de verificar as principais funcionalidades da ferramenta na versão final.
            \end{itemize}
    \section{Contribuições Esperadas}
        \label{cap:introducao:sec:contribuicoes}
        % backlink - resumo <#ID_491307769>
        Após a finalização deste trabalho, deseja-se
        % Ferramenta
            % elimitar ou diminuir a dependência de outras ferramentas
            uma maior independência para a JGOOSE e, consequentemente, seus usuários,
            % Maior facilidade no uso da ferramenta JGOOSE NOOP
            além de aumentar o destaque na comunidade e promover a adoção da ferramenta para fins de modelagem i*.
        % Eng de Requisitos
            % melhor rastreamento dos requisitos ?
        % Reconhecimento
            % Comunidade i* - i* wiki
            Por fim, espera-se uma contribuição significativa diante das ferramentas da comunidade i*,
            trazendo um reconhecimento para o grupo de pesquisa do LES e todos os envolvidos no desenvolvimento e progresso da JGOOSE.
            Além disso, seria uma nova ferramenta no quadro comparativo do i* Wiki \cite{site2013iwiki}.
    \section{Estrutura do Trabalho}
        \label{cap:introducao:sec:organizacao}
        Basicamente,
        o restante deste trabalho encontra-se organizado da seguinte maneira:
            Nos capítulos \ref{cap:framework} e \ref{cap:jgoose} são apresentados alguns fundamentos teóricos necessários para uma melhor compreensão da área de estudo de que trata este trabalho.  No capítulo \ref{cap:framework}, os conceitos básicos e as características do \emph{Framework} i*, bem como suas variações, são apresentados.  Já no capítulo \ref{cap:jgoose}, após um estudo sobre a evolução histórica do software JGOOSE, uma análise e discussão detalhada de sua arquitetura, na versão 2013, é realizada.
            No capítulo \ref{cap:proposta}, a proposta é detalhada através de uma visão geral do projeto e arquitetura da nova interface. Também é apresentada uma discussão sobre os principais recursos disponíveis aos usuários.
            Em seguida, no capítulo \ref{cap:estudo-de-caso} é apresentado um estudo de caso, mostrando e avaliando a aplicação da ferramenta em um domínio específico.
            Finalmente, o capítulo \ref{cap:conclusao} reúne as análises e considerações finais sobre os resultados, bem como relata sobre os possíveis trabalhos futuros.

        Outra composição importante, do ponto de vista técnico-computacional, é o apêndice \ref{apendice:istarml}: uma documentação sobre a API desenvolvida durante a fase de implementação.
        
% Framework i*
\chapter{Framework i*, Variações e Ferramentas}
    \label{cap:framework}
    % intro
        Neste capítulo são apresentados os conceitos básicos necessários para o entendimento sobre a técnica de modelagem organizacional do framework i*, bem como os trabalhos baseados nessa técnica.
        Por fim, discute-se sobre algumas ferramentas de modelagem i* e/ou variações.
    % topics
        % framework i*
            Inicialmente, na seção \ref{cap:framework-sec:istar}, são apresentados os conceitos fundamentais do i* através dos seus dois componentes de modelagem e alguns dos meta-modelos existentes.
        % variacoes
            Na seção \ref{cap:framework-sec:variacoes}, mostra-se algumas variações ou extensões da proposta inicial do i* em \cite{yu1995modelling}.
        % ferramentas
            Em seguida, na seção \ref{cap:framework-sec:ferramentas}, comenta-se sobre a ferramentas OME, uma ferramenta que geram arquivos de entrada para a JGOOSE.
        % considerações finais do capítulo
            Por fim, na seção \ref{cap:framework-sec:conclusao}, são feitas algumas considerações finais do capítulo.
    %
    \section{O Framework i*}
        \label{cap:framework-sec:istar}
        % intro
            % o que é ?
                O framework i* (pronunciado "i-star"
                        \footnote{O nome i*, pronunciado em inglês "i-star" faz referência ao conceito sobre uma intencionalidade distribuída. No Brasil, são comuns as pronúncias "i-estrela" e "i-star".}),
                    originalmente proposto por Yu \cite{yu1995modelling},
                é um framework de modelagem organizacional conceitual.
                Ou seja, ajuda no desenvolvimento de modelos que auxiliam a análise de sistemas sob uma visão estratégica e intencional de processos que envolvem vários participantes.

            % pra que server?
                O framework i* preocupa-se principalmente com a análise do contexto organizacional e social de um sistema.
                O sistema, nesse caso, não consiste somente em componentes técnicos, mas também de elementos humanos.
            % aplicações em várias áreas
                Como o framework i* é bastante flexível para representar situações envolvendo interações entre múltiplos participantes,
                esse framework pode ser utilizado para representar variados contextos organizacionais.

                A seguir, têm-se alguns contextos onde a modelagem i* vem sendo aplicada:
                    \begin{itemize}
                        \item[] \textbf{Engenharia de Requisitos}:
                            é uma das áreas de aplicações mais comuns do i*, principalmente nas fases iniciais do processo de engenharia de requisitos (\emph{Early Requirements})
                            \cite{yu1997towards} e \cite{maiden2004model};

                        \item[] \textbf{Modelagem de Negócio (Business Modeling)}:
                            estudos na área apresentaram o uso do i* para visualização explícita da intencionalidade por trás dos processos de negócios.
                            Isso ajuda a se obter um melhor entendimento sobre o trabalho, além de facilitar seu replanejamento
                            \cite{yu1996models} \cite{kolp2003organizational};

                        \item[] \textbf{Desenvolvimento Orientado à Objeto}:
                            em \cite{castro2000closing} e \cite{castro2001integrating}
                            utilizou-se da pUML (precise UML) \cite{evans1999core}
                            e da \emph{Object Constraint Language} (OCL) \cite{warmer2003object}
                            para tratar dos requisitos finais (\emph{Late Requirements}), além de usar o framework i* para os requisitos iniciais;

                        \item[] \textbf{Desenvolvimento Orientado à Agentes}:
                            em \cite{bresciani2004tropos} apresentou-se o uso de agentes com estrutura BDI (\emph{Believe, Desire and Intention}) \cite{rao1995bdi} para realizar análises na fase inicial de requisitos.
                            Já em \cite{bastos2004enhancing}, utilizou-se de Sistemas Multi-Agentes (SMA) para especificar a estrutura organizacional;

                        \item[] \textbf{Segurança, Confiabilidade e Privacidade}:
                            a modelagem i* pode ajudar a lidar com elementos de segurança, confiabilidade e privacidade, através do estudo dos conflitos de intenções de diferentes entidades sociais \cite{yu2001modelling};
                    \end{itemize}

            % orientado a agente/objetivo?
                Segundo \cite{yu2011social}, pode-se dizer que o i* é um framework de modelagem tanto orientado a agentes quanto orientado a objetivos, pois sua essência é realizada na combinação de agentes/atores e objetivos.
                Ambos os paradigmas, Orientação à Agentes \cite{mao2005organizational} e Orientação à Objetivos \cite{van2004goal}, têm apresentado bons resultados em contextos de modelagem organizacional, principalmente em modelagens da fase inicial (\emph{Early Requirements}) do processo de engenharia de requisitos. % TODO: cite?

            % como funciona?
                O i* é composto por dois components de modelagem: modelo de dependências estratégicas e modelo de razões estratégicas.
                Esses componentes auxiliam na representação, respectivamente, das dependências entre atores e dos detalhes por atrás das dependências de cada ator.
                É fundamental conhecer as notações e saber aplicar esses conceitos para se construir um bom modelo organizacional \cite{webster2005survey}.
                A seguir, são apresentados os conceitos e notações por trás dos modelos \cite{site2013iwiki}.

        \subsection{Modelo de Dependências Estratégicas}
            % intro
                O modelo de Dependência Estratégica (SD)\footnote{SD, do inglês Strategic Dependency},
                representa um conjunto de relacionamento estratégicos externos entre os atores organizacionais, formando uma rede de dependências.
                Fornece uma visão mais abstrata e ampla da organização, sem se preocupar com os detalhes (razões internas) por trás dessas dependências.

            \paragraph{Atores, Especializações e Fronteira}
                \begin{enumerate}[i.] % for capital roman numbers.
                    \item \textbf{Ator} pode ser definido como uma entidade (humana ou computacional) que age sobre o meio que está inserido para conquistar seus objetivos, exercitando seu \emph{know-how} \cite{yu1995modelling}. Atores podem ser vistos como uma referência genérica a qualquer unidade que se possa atribuir dependências intencionais. Os atores possuem relações de dependências com outros atores para um determinado fim. Quando existe uma necessidade de maiores detalhes sobre um modelo organizacional, atores podem ser diferenciados em três especializações: agentes, posições e papéis. A Figura \ref{fig:atores} exemplifica os tipos de atores, enquanto a Figura \ref{fig:atores-exemplo} apresenta um exemplo dos possíveis relacionamento entre os tipos de atores. A seguir, descreve-se esses tipos:
                    
                    \begin{figure}[h!]
                        \centering
                            \includegraphics[scale=0.8]{Figuras/istar/atores.jpg}
                            \caption{Exemplo de relações entre atores}
                            \label{fig:atores-exemplo}
                    \end{figure}

                    \begin{figure}[h!]
                        \centering
                            \subfigure[fig:ator:a][Ator]{\includegraphics[scale=0.8]{Figuras/istar/ator.jpg}}
                            \subfigure[fig:ator:b][Ator]{\includegraphics[scale=0.8]{Figuras/istar/ator-agente.jpg}}
                            \subfigure[fig:ator:c][Ator]{\includegraphics[scale=0.8]{Figuras/istar/ator-posicao.jpg}}
                            \subfigure[fig:ator:d][Ator]{\includegraphics[scale=0.8]{Figuras/istar/ator-papel.jpg}}
                            \caption{Notação de Ator (a), Agente (b), Posição (c) e Papel (d).}
                            \label{fig:atores}
                    \end{figure}
                    
                    %
                    \item \textbf{Agente} é a decomposição de um ator que possui manifestações físicas concretas. Refere-se tanto a humanos quanto a agentes de software ou hardware. Um agente possui dependências independentemente do papel que está executando. As características de um agente normalmente não são fáceis de se transferir para outros atores/agentes. São como experiências, habilidades ou, até mesmo, limitações físicas.
                    %
                    \item \textbf{Papel} é a caracterização abstrata do comportamento de um ator dentro de determinados contextos sociais ou domínio de informação. Essas características devem ser facilmente transferíveis a outro ator social. As dependências associadas a um papel são aplicáveis independentemente do agente que desempenha o papel.
                    %
                    \item \textbf{Posição} representa uma abstração intermediária entre um agente e um papel. É o conjunto de papéis tipicamente executados por um agente, ou seja, representa uma posição dentro da organização onde o agente pode desempenhar várias funções (papéis). Diz-se que um agente ocupa uma posição e uma posição cobre um papel.
                \end{enumerate}

            % actors associations
            \paragraph{Associações entre Atores:}
                As associações entre os atores são descritas através de links de associação (conforme a Figura \ref{fig:associations}.
                Essas associações podem ser de seis tipos:

                \begin{enumerate}[i.]
                    \item \textbf{\emph{IS PART OF}} (faz parte de) - Nessa associação cada papel, posição e agente pode ter sub-partes. Em \emph{IS PART OF} há dependências intencionais entre o todo e sua parte. Por exemplo, a dependência do todo sobre suas partes para manter a unidade na organização.

                    \item \textbf{\emph{ISA}} (é um) - Essa associação representa uma generalização, com um ator sendo um caso especializado de outro ator. Ambas, \emph{ISA} e \emph{IS PART OF}, podem ser aplicadas entre quaisquer duas instâncias do mesmo tipo de ator.
                    
                    \item \textbf{\emph{PLAYS}} (executa) - A associação plays é usada entre um agente e um papel, com um agente executando um papel. A identidade do agente que executa um papel não deverá ter efeito algum nas responsabilidades do papel ao qual está associado, e similarmente, aspectos de um agente deverão permanecer inalterados mesmo associados a um papel que este desempenha.
                    
                    \item \textbf{\emph{COVERS}} (cobre) - A associação covers é usada para descrever uma relação entre uma posição e os papéis que esta cobre.
                    
                    \item \textbf{\emph{OCCUPIES}} (ocupa) - Esta associação é usada para mostrar que um agente ocupa uma posição, ou seja, o ator executa todos os papéis que são cobertos pela posição que ele ocupa.
                    
                    \item \textbf{INS} - Esta associação é usada para representar uma \textbf{INS}tância específica de uma entidade mais geral. Por exemplo, quando se deseja representar um agente que é uma instanciação de outro agente.
                \end{enumerate}
                \begin{figure}[h!]
                    \centering
                        \includegraphics[width=0.8\linewidth]{Figuras/istar/actor-associations.jpg}
                        \caption{Tipos de associações entre atores \cite{site2013iwiki}.}
                        \label{fig:associations}
                \end{figure}

            % Elements
            \paragraph{Relação de Dependência}
                Uma relação de dependência pode ser deifinida como um acordo entre dois atores.
                Os elementos que compõe uma relação de dependência são:
                \begin{enumerate}[i.]
                    \item \emph{\textbf{Depender}}: é o ator dependente, ou seja, o ator que precisa que o acordo (\emph{Dependum}) seja realizado. Esse ator não se importa como o outro ator (\emph{Dependee}) irá satisfazer a necessidade da dependência.
                    \item \emph{\textbf{Dependum}}: é o elemento intermediário, objeto de questionamento e validação, da relação de dependência. % Ou seja, TODO
                    \item \emph{\textbf{Dependee}}: é o ator que tem a responsabilidade de satisfazer a relação de dependência.
                \end{enumerate}

                Dessa forma, pode-se classificar o tipo de uma relação de dependência com base em dos seguintes tipos de \emph{Dependum}:
                \begin{enumerate}[i.]
                    \item \textbf{Objetivo} (\emph{Goal}) - é uma declaração de afirmação sobre um certo estado do mundo. Deve ser de fácil verificação. O \emph{Dependee} é livre para tomar qualquer decisão para satisfazer o objetivo e é esperado que ele o faça. Não importa para o \emph{Depender} como o \emph{Dependee} irá alcançar esse objetivo.
                    %
                    \item \textbf{Tarefa} (\emph{Task}) - é uma atividade a ser realizada pelo \emph{Dependee}. Tarefas podem ser vistas com a realização de operações, processos e etc. Porém, não deve ser uma descrição passo-a-passo ou uma especificação completa de execução de uma rotina.
                    %
                    \item \textbf{Recurso} (\emph{Resource}) - é entidade (física ou informativa) a ser entregue para o \emph{Depender} pelo \emph{Dependee}. Satisfazendo-se esta dependência, o \emph{Depender} está habilitado a usar essa entidade como um recurso.
                    %
                    \item \textbf{Objetivo-Soft} (\emph{Softgoal}) - é semelhante ao Objetivo, porém os critérios de avaliação e verificação são mais subjetivos. O \emph{Depender} pode decidir sobre o que constitui a realização satisfatória do objetivo.
                \end{enumerate}
                % \begin{figure}[h!]
                %     \centering
                %         \includegraphics[width=0.7\linewidth]{Figuras/istar/dependency-links.jpg}
                %         \caption{Exemplo de Relação de Dependência (\emph{Dependder} -> \emph{Dependum} -> \emph{Dependee})}
                %         \label{fig:dependency-links}
                % \end{figure}
                % A Figura \ref{fig:dependency-links} apresenta alguns exemplos de relações de dependências.
            
            % Links (One side)
            \paragraph{Ligação de dependência}
                É uma estritamente a conexão entre os elementos de forma direcionada.
                Assim, pode-se ter somente duas opções de conexão: início no \emph{Depender}, fim no \emph{Dependum} e início no \emph{Dependum} e fim no \emph{Dependee}. Essa conexão é definida por um segmento contínuo, com a letra "D" sobrescrita, e direcionada da origem para o destino (conforme os exemplos da Figura \ref{fig:dependency}).
                % Assim, pode-se pensar na seguinte estrutura de nodos e ligações:
                %     1. \emph{Depender} (nodo); 2. Ligação de dependência (ligação); 3. \emph{Dependum} (nodo); 4. Ligação de dependência (ligação); 5. \emph{Dependee} (nodo).
                \begin{figure}[h!]
                    \centering
                        \subfigure[fig:dependency:empty][Ligação de Dependência.]{\includegraphics[scale=1]{Figuras/istar/dependency-empty.jpg}}
                        \\
                        \subfigure[fig:dependency:depender-dependum][Ligação de Dependência: do \emph{Depender} para o \emph{Dependum}.]{\includegraphics[width=0.4\linewidth]{Figuras/istar/dependency-depender-dependum.jpg}}
                        ~
                        \subfigure[fig:dependency:dependum-dependee][Ligação de Dependência: do \emph{Dependum} para o \emph{Dependee}.]{\includegraphics[width=0.4\linewidth]{Figuras/istar/dependency-dependum-dependee.jpg}}
                        \caption{Exemplos de ligação de dependência.}
                        \label{fig:dependency}
                \end{figure}
                
        % [end subsection]
        \subsection{Modelos de Razões Estratégicas (SR)}
            % SR
                Já o modelo de Razões Estratégicas(SR)\footnote{SR, do inglês Strategic Rationale},
                representa os detalhes das razões internas que estão por trás das dependências dos atores.
                Com isso, é possível detalhar os interesses, preocupações e motivações específicas de um ator.
                Esse tipo de modelo também torna possível a avaliação de alteranativas em definições de processos.

                O modelo SR, além de poder conter todos os atores e dependências do SD, "abre-se" os Atores para se detalhar as dependências e expressar as razões.
                Ou seja, para os atores que precisam ser detalhados, é habilitado o limite da fronteira que deve estar visível e com espaço o suficiente para receber os elementos de dependência e/ou ligações internas. 
                Dessa forma, pode-se pensar nos elementos internos ao ator, ou seja, dentro da área de fronteira, como "pertencentes" ao ator.
                A seguir, são descritos os demais elementos de um modelo SR.

            \paragraph{Fronteira}
                Uma fronteira indica os limites intecionais de um determinado ator. Todos os elementos dentro dos limites de um ator, são explicitamente desejos ou pretenções desse ator. Uma fronteira é representada por um círculo tracejado e o elemento do ator dessa fronteira deve ser sobreposto a ela, ficando acima do tracejado (conforme a Figura \ref{fig:boundary}).

                \begin{figure}[h!]
                    \centering
                        \subfigure[fig:boundary:a][Fronteira Vazia]{\includegraphics[width=0.3\linewidth]{Figuras/istar/boundary-empty.jpg}}
                        \subfigure[fig:boundary:b][Fronteira com elementos internos]{\includegraphics[width=0.3\linewidth]{Figuras/istar/boundary-non-empty.jpg}}
                        \caption{Exemplos de fronteira do ator.}
                        \label{fig:boundary}
                \end{figure}

            \paragraph{Ligação de meio-fim (\emph{means-end})}
                    % TODO: review.
                    É representada graficamente por uma seta direcionada ao nó fim, significando o meio para atingir um fim (objetivo, recurso, \emph{softgoal}, ou uma tarefa). A Figura \ref{fig:means-end} exemplifica este tipo de ligação.
                    \begin{figure}[h!]
                        \centering
                            \includegraphics[scale=1]{Figuras/istar/means-end.jpg}
                            \caption{Exemplo de ligação meio-fim.}
                            \label{fig:means-end}
                    \end{figure}
                    %
                \paragraph{Ligação de decomposição (\emph{decomposition}):}
                    É responsável por detalhar e expressar da melhor como realizar uma determinada tarefa, através da decomposição em sub-elementos ligados  a  tarefa  principal (superior)  através  de  um  segmento  de  reta cortado.  Esses sub-elementos  podem  ser:  metas,  tarefas,  recursos  e  objetivos-soft.
                    \begin{figure}[h!]
                        \centering
                            \includegraphics[scale=1]{Figuras/istar/decomposition.jpg}
                            \caption{Exemplo de ligação de decomposição.}
                            \label{fig:decomposition}
                    \end{figure}
                    %
                \paragraph{Ligações de Contribuição (\emph{contribution}):}
                    As ligações de contribuição são para ligar elementos à exclusivamente um objetivo-soft (\emph{softgoal}).
                    Essa ligação ajuda a modelar a forma como os elementos contribuem para a satisfação desse objetivo-soft (\emph{softgoal}).
                    Essas ligações de contribuição, ilustradas na Figura \ref{fig:contributions}, podem ser:
                    \begin{figure}[h!]
                        \centering
                            \includegraphics[scale=1]{Figuras/istar/contributions.jpg}
                            \caption{Ligações de contribuição \cite{site2013iwiki}.}
                            \label{fig:contributions}
                    \end{figure}
                    \begin{enumerate}[i.]
                        \item \textbf{\emph{Make}}: é uma contribuição positiva, suficientemente forte para satisfazer o objetivo-soft. 
                        \item \textbf{\emph{Some +}}: é uma contribuição positiva, mas cuja força de influência é desconhecida. Pode equivaler a um \emph{make} ou a um \emph{help}.
                        \item \textbf{\emph{Help}}: é uma contribuição positiva fraca, pois não é suficiente para que ela sozinha satisfaça o objetivo-soft.
                        \item \textbf{\emph{Unknown}}: é uma contribuição cuja influência é desconhecida. 
                        \item \textbf{\emph{Hurt}}: é uma contribuição negativa fraca, porém não é suficiente para que ela sozinha recuse a  satisfação de um objetivo-soft.
                        \item \textbf{\emph{Some -}}: é uma contribuição negativa, mas a força de sua influência é desconhecida. Pode equivaler a um \emph{hurt} ou a um \emph{break}.
                        \item \textbf{\emph{Break}}: é uma contribuição negativa, suficientemente forte para rejeitar a satisfação do objetivo-soft.
                        \item \textbf{\emph{Or}}: é uma contribuição onde o objetivo-soft é satisfeito se algum dos descendentes for satisfeitos. 
                        \item \textbf{\emph{And}}: é uma contribuição onde o objetivo-soft é satisfeito se todos os descendentes forem satisfeitos.
                    \end{enumerate}
                
        % [end subsection]
        \subsection{iStarML}
            % intro
                Muitas ferramentas foram criadas com base nos conceitos do framework i* ou de variações desse \cite{cares2012towards} \cite{cares2011towards}.
                Isso acabou gerando modelos específicos para cada ferramenta, dificultando o intercâmbio de modelos entre essas ferramentas.
                Pensando nisso, foi desenvolvida a iStarML.
                Um meta-modelo baseado em XML (\emph{Extensible Markup Language}) usado para representar modelos i* \cite{cares2008istarml}.
                
                O principal objetivo desse meta-modelo é proporcionar um formato de intercâmbio entre os outros formatos de modelos do i*.
                Ou seja, a especificação iStarML deve suportar todas as outras definições e especificações dos modelos já propostos.
                Com isso, tudo o que se consegue especificar no formato TELOS, por exemplo, deve-se conseguir também no formato iStarML.

                Como prova de conceitos, foi realizado um estudo onde se aplicou o iStarML estritamente para fazer a interconexão entre duas ferramentas diferentes \cite{colomer2011model}.
                Nesse estudo, as ferramentas aplicadas foram jUCMNav \cite{kealey2006integrating} e a HiME \cite{lopez2009hime}.

                Porém, existe a preocupação sobre a real adoção da especificação iStarML. Como exemplo, tem-se a promessa da ferramenta OpenOME, que diz estar trabalhando para implementar rotinas de importação/exportação em iStarML \cite{horkoff2011openome} \cite{laue2011adding}.

                Conforme já mencionado no Capítulo \ref{cap:introducao}, é objetivo dessa pesquisa a implementação e adoção do iStarML como especificação do formato de arquivo padrão. Portanto, detalhes mais específicos sobre a linguagem iStarML serão feitas no Capítulo \ref{cap:proposta}.

        % [end subsection]
    % [end section]
    \section{Variações Baseadas no Framework i*}
        \label{cap:framework-sec:variacoes}
        % ITU-T = GRL + UCM.
        % yu95, i* wiki, GRL, Tropos, Aspectual i*, i*-c [6]
        % intro
            % Estudos da área já realizaram comparações entre as ferramentas.
            % Diferentes grupos de pesquisas que adotaram o framework i*.
            % Por vezes, o framework i* não atende todas as necessidades de um determinado grupo de pesquisadores.

        % exemplos
            % "O framework i* foi inicialmente proposto por [Yu 1995], mas hoje existem algumas extensões ou variações para sua versão original"
            % "Essas variações surgiram de diferentes grupos de pesquisa para atender ao propósito particular de cada um deles, e com isso surgiram diversas ferramentas de suporte."

        %problemas das variações i*
            % Conforme \cite{lucena2008}, as divergências quanto ao uso do i* podem causar:
            %     \begin{itemize}
            %         \item Divisão do esforço, ao passo que cada grupo de pesquisa irá focar no desenvolvimento de ferramentas de suporte ao seu próprio i*;
            %         \item Erros na semântica entre os projetistas e os leitores de um modelo particular do i*;
            %         \item Inibição da adoção/uso do i* por parte de novos usuários.
            %     \end{itemize}

        % vários estudos comparativos já foram realizados
            %http://istarwiki.org/tiki-index.php?page=i%2A+Modelling+Techniques&structure=i%2A+Wiki+Home
        % [end subsection]
        \subsection{i* Wiki}
            % o que é?
                O i* Wiki, é um projeto criado com o intuito de reunir trabalhos relativos ao i*, de forma colaborativa
                    \cite{site2013iwiki} \cite{leuf2001wiki}.
                Com isso, a comunidade incentiva a colaboração dos usuários do framework, por meio de \emph{feedback} ou mesmo inserção de conteúdo em site oficial.
                Além disso, esses usuários podem sugerir alternativas ou extensões sintáticas e semânticas em relação a linguagem utilizada.
            
            % como funciona?
                Apesar da ampla visão que a comunidade pode ter com os trabalhos divulgados no site, a intenção é fornecer e evoluir uma única versão semântica do i*.
                Dessa forma, o i* Wiki funciona sobre duas versões do guia para o i* \cite{site2013iwiki}:
                    uma versão estável, servindo de referência para os usuários;
                    outra versão aberta a discussão, acessível aos usuários registrados no site e passível de comentários e sugestões individuais.
                Além disso, o site reúne um conjunto de Estudos de Casos, Publicações e Eventos relacionados a área de i*.
        % [end subsection]

        \subsection{\emph{Goal-Oriented Requirements Language} (GRL)}
            A GRL é uma linguagem de apoio à modelagem orientada a agentes e objetivos.
            Assim como o i*, a GRL foca na modelagem dos relacionamentos estratégicos entre atores e seus objetivos.
            Pode-se pensar como uma alternativa que concentra recursos das metodologias NFR (\emph{Non-Functional Requirements}), i* e Tropos \cite{regev2005goals}.
            Outro ponto interessante é que a GRL é escalável, podendo se trabalhar com diferentes níveis de granularidade, em múltiplos diagramas ou visões de um mesmo modelo.

            Além disso, uma combinação da GRL com a \emph{Use Case Map} (UCM) deu origem a \emph{User Requirement Notation} (URN) - um padrão internacional do \emph{International  Telecommunication  Union} (ITU) para notação de requisitos de usuário.

        % [end subsection]

        \subsection{Tropos}
            Tropos é um projeto que foi lançado em 2000 \cite{mylopoulos2001uml}, e visa apoiar a construção de sistemas de software orientados a agentes.
            O projeto reúne um grupo de autores de diversas Universidades no Brasil, Canadá, Bélgica, Alemanha, Itália, etc.
                % Tropos tem como objetivo o  desenvolvimento de sistemas de acordo com as reais necessidades de uma organização, buscando um melhor casamento entre o sistema e o ambiente em constante mudança.
            O processo de desenvolvimento segundo esta metodologia inicia com um estudo e elaboração de um modelo do ambiente no qual o sistema em desenvolvimento irá operar.
            Este modelo é refinado até que este represente o ambiente com o sistema em seu contexto.
            Cada modelo é descrito em termos dos atores observados no ambiente em modelagem, seus objetivos e relacionamentos.
            A metodologia Tropos oferece um framework que engloba as principais fases de desenvolvimento de software, com o apoio das seguintes atividades: Requisitos Iniciais, Requisitos Finais, Projeto Arquitetural e Projeto Detalhado.
        % [end subsection]

    % [end section]
    \section{Ferramenta OME }
        \label{cap:framework-sec:ferramentas}
        % intro
            Atualmente, existem várias ferramentas de modelagem i*.
            Pode-se encontrar mais de 20 ferramentas referenciadas no site do i* Wiki \cite{site2013iwiki}.
            % Ferramentas já foram comparadas. Listar trabalhos.
            Além disso, alguns trabalhos já realizaram comparações sobre ferramentas do framework i*,
                como em \cite{santos2008istar}(Tabela 6) e no site \cite{site2013iwiki}.

        % Pq só as duas?
            % Conforme foi comentado no Capítulo \ref{cap:introducao}, os arquivos de entrada aceitos pelo JGOOSE devem estar no formato TELOS (.tel).
            % Além disso, conforme a apresentação da primeira versão da ferramenta em \cite{vicente2006}, as duas ferramentas que trabalham com esse formato são a OME e a OpenOME.
            % call next
            % Dessa forma, a OME e a OpenOME são apresentadas a seguir.

        % \subsection{OME}
            % intro
                O Ambiente de Modelagem Organizacional, tradução literal de OME - Organization Modelling Environment, é um editor gráfico de propósito geral para dar suporte à modelagem orientada a objetivo e/ou orientada a agentes.
                É uma aplicação Java para \emph{desktop} desenvolvida na Universidade de Toronto \cite{ome2013}.

            % about
                A ferramenta possui recursos que auxiliam o usuário no desenvolvimento e manipulação de modelos i* e NFR (\emph{Non-Functional Requirements}) \cite{chung2000non}.
                % history
                Em 2004, o desenvolvimento foi parado e seu código foi portado para a plataforma Eclipse \cite{eclipse}, dando origem a sua versão em código aberto chamda OpenOME \cite{horkoff2011openome}.
                % Aparentemente, a ferramenta parou na versão 3 (ou OME3) - mas ainda é mantida no site http://www.cs.toronto.edu/km/ome/.

                Apesar do seu desenvolvimento ter sido finalizado, ainda existem usuários da OME3.
                Além disso, a ferramenta possui um manual do usuário online (http://www.cs.toronto.edu/km/ome/docs/manual/manual.html) e é de fácil utilização. A maioria dos recursos i*, por exemplo, estão de acordo com \cite{yu1995modelling}.

            % exemplo final de tela:
                \begin{figure}[h!]
                    \centering
                        \includegraphics[width=0.8\linewidth]{Figuras/istar/tela-ome.jpg}
                        \caption{Exemplo de modelagem com a ferramenta OME3 (arquivo "Meeting-Schedule.tel")}
                        \label{fig:tela-ome}
                \end{figure}

                Como exemplo, têm-se na Figura \ref{fig:tela-ome} um dos modelos de exemplos já contidos na ferramenta OME3 junto à instalação.
                Nessa tela, pode-se observar a barra de ferramentas (\emph{toolbar}) com as opções para criação dos elementos i*: atores, elementos de dependência, links  de  dependência, links  de  associação, etc.

                Cabe relembrar que essa ferramenta gera os modelos i* no formato TELOS, atualmente aceitos pela ferramenta JGOOSE.
                Além disso, considerando que a ferramenta apresenta problemas de instabilidade e seu processo de desenvolvimento foi descontinuado, acredita-se que o JGOOSE deva adotar uma nova solução para este fim.

    % [end section]
    
    \section{Considerações Finais do Capítulo}
        \label{cap:framework-sec:conclusao}
        % o framework i* é importante (viu-se aplicações e benefícios em diversas áreas)
        %
            Neste Capítulo foram apresentados
                os conceitos gerais do framework i*, bem como as notaçãos da última versão do i* Wiki.
            Além disso,
                as variações do i* e seus respectivos meta-modelos, são estudados e analisados.
            Apesar das variações apresentadas, destaca-se a solução de intercâmbio entre diferentes formatos e ferramentas, iStarML.
            O iStarML foi adotado pelo editor proposto e será discutido melhor no Capítulo \ref{cap:proposta}.
    % [end section]

% bibliography, just for auto-link in sublime text
% before compile main.tex, comment line below
% \bibliography{ref-unioeste,ref-commons,ref-books,ref-tecnologias,ref-istar}

% Capítulo 3
    \chapter{JGOOSE}
        \label{cap:jgoose}
        % intro
            Neste capítulo,
            é apresentada a ferramenta JGOOSE (\emph{Java Goal into Object Oriented Standard Extension}), bem como uma breve análise das versões ao longo dos anos.
            Além disso, é analisada a organização arquitetural e o projeto da ferramenta na versão 2013 para posteriormente, no Capítulo \ref{cap:proposta}, realizar a integração com o Editor proposto neste trabalho.
        %
    \section{Visão Geral}
        % intro
            A ferramenta JGOOSE tem como objetivo geral o mapeamento de modelos organizacionais i* para casos de uso UML \cite{vicente2006}.
            A ferramenta trabalha sobre as diretrizes propostas por \cite{santander2002integrando} e é com base nessas diretrizes que a ferramenta interpreta os modelos organizacionais do framework i* e apresenta os casos de uso mapeados no formato proposto por \cite{cockburn2001writing}
        % 
    \section{Histórico de Versões}
        % intro
            A JGOOSE, desde a sua primeira versão feita por \cite{vicente2006}, já passou por várias melhorias e aprimoramentos.
            A seguir, comenta-se sobre algumas considerações a respeitdo das versões do JGOOSE.
        % 
        \begin{itemize}
            \item \textbf{Antecedentes} - a ferramenta JGOOSE algumas influências da área de mapeamento de modelos i* para UML \cite{pedroza2004ferramentas}. Porém, com os avanços das tecnologias, os produtos de softwares também precisam ser atualizados.
            Pensando nisso, foi proposto uma nova ferramenta, a JGOOSE, mas que mantivesse as funcionalidades principais.

            \item \textbf{Versão 2006}, por Vicente \cite{vicente2006}: nova implementação em Java; importação de arquivos em TELOS; 

            \item \textbf{Versão 2011}, por Bischke \cite{brischke2011melhorando}: implementação de três diretrizes faltantes; implementação da exportação em XMI; refinamento manual de caso de uso; visualizador de caso de uso;

            \item \textbf{Versão 2013}, por Peliser \cite{peliser2013aprimorando}: correção de \emph{bugs} na aplicação das diretrizes; otimizações de código (algoritmos); melhorias na interface gráfica; melhorias na documentação.
        \end{itemize}
        % 
        % finalizacao comparativa
        Com base nesse apanhado histórico, percebe-se que a ferramenta caminha em direção ao profissionalismo e qualidade na área de mapeamento de modelos i* para caso de uso.
        % 
    % 
    \section{Projeto e Arquitetura}
        \begin{figure}[htb]
            \centering
                \includegraphics[width=0.7\linewidth]{Figuras/jgoose-arquitetura.jpg}
                \caption{Arquitetura e Fluxo de processamento da JGOOSE.}
                \label{fig:jgoose-arquitetura}
        \end{figure}
        Conforme a seção anterior, a ferramenta passou por várias mudanças. Porém, seu fluxo de processamento manteve-se de acordo com os passos e diretrizes propostos por \cite{santander2002integrando}.
        
    %
    % \section{Código Fonte}
    %     Nessa seção, é realizada algumas discussões sobre o código fonte da ferramenta.

    % \section{Contribuções da Ferramenta}
    \section{Considerações Finais do Capítulo}
        O entendimento da ferramenta JGOOSE se faz necessário visto o objetivo de se incorporar um ambiente de edição de modelos organizacionais i*.
        Assim, a proposta deste trabalho trata especificamente da etapa (1) "Capturar Tokens Específicos" - passando a suportar o formato de arquivo iStarML.
        % Com algumas análises do código fonte, 
        % Contribuição literária-científica (publicações)
        % Contribuição educacional: alunos usando e desenvolvendo % auxílio no ensino (uso) e na aprendizagem (dev)


% bibliography, just for auto-link in sublime text
% before compile main.tex, comment line below
% \bibliography{ref-unioeste,ref-commons,ref-books,ref-tecnologias,ref-istar}

% 
% Escolhas e JUSTIFICATIVAS!?
% 
\chapter{E4J - Editor i* para JGOOSE}
    \label{cap:proposta}
    % intro
        Este capítulo apresenta a solução computacional \ferramenta{}, um editor de modelos organizacionais i* integrado à ferramenta JGOOSE.

        Para a apresentação desta proposta, este capítulo foi dividido nas seguintes seções:
            \begin{itemize}
                \item ... % TODO
            \end{itemize}
    % [end]

    \section{Visão Geral}
        \label{cap:proposta-sec:overview}
        % tópico - intro

        % nesse sentido,
        %   adotou-se 
        A E4J (pronunciado ``i-fór-jey''), representa a abreviação de ``Editor para JGOOSE'', mais especificamente: \textbf{E}ditor i* \textbf{para} \textbf{J}GOOSE.


        Para se usar com efetividade a \ferramenta{},
            é necessário conhecer os conceitos originais da modelagem organizacional i* propostos por Yu \cite{yu1995modelling}, vistos no capítulo \ref{cap:framework} deste trabalho.
            Também é necessário conhecer o funcionamento da ferramenta JGOOSE, visto no capítulo \ref{cap:jgoose}, para entender o mapeamento dos casos de uso UML.

        % \subsection{Considerações Técnicas}
        %     A E4J foi desenvolvida na linguagem de programação JAVA \cite{java}.
        %         % A JGOOSE ja é java,
        %         % ...
        %         % vantagens e beneficios de se manter a Linguagem JAVA

        \subsection{JGraphX}
            A JGraphX é uma Biblioteca para Java Swing (\emph{Java Swing Library}) da \emph{mxGraph} \cite{mxgraph}.
                Consiste em um conjunto de estruturas e funcionalidades que facilitam a produção de Aplicações Java Swing \cite{javaswing}.
                Com essa biblioteca é possível criar aplicações interativas voltadas principalmente para manipulação de diagramas.

            O núcleo dessa biblioteca é fundamentada na Teoria dos Grafos \cite{teoria dos grafos}.
                Um grafo consiste em um conjunto de vertices, chamados também de nodos ou nós, e arestas.
                As arestas são conexões entre os nós.
                Na estrutura da JGraphX, existe também o conceito de célula, que representa um elemento do grafo: arestas, vértices ou um grupo destes.

                Essa biblioteca foi escolhida pelos seguintes fatores:
                    \begin{itemize}
                        \item é uma biblioteca Java focada em aplicações para manipulação de diagramas sobre a estrutura de grafos. Como os modelos organizacionais i* podem ser construídas sobre estruturas de grafos, este ítem é relativamente significante para a escolha da biblioteca.
                        \item boa representatividade na comunidade. Essa biblioteca é citada em diversos fóruns da área e, recentemente, foi realizado um estudo comparativo entre a biblioteca JGraphX e a JUNG, ambas soluções para visualização e manipulação de grafos \cite{souza2013analise}. Nesse estudo, segundo os critérios avaliados, a JGraphX apresentou melhores caraterísticas que a JUNG.
                        \item está sob a licensa BSD \cite{bsd}.
                        \item é disponibilizada em repositório Maven com boa documentação (JavaDoc) e código-fonte.
                    \end{itemize}
    % [end]

    \section{Projeto e Arquitetura}
        \label{cap:proposta-sec:design}
        % visões: SD, SR; DFD; Camadas; Componentes;
            Antes de começar a desenvolver efetivamente a \ferramenta{},
                foram desenvolvidos alguns modelos que auxiliaram o entendimento das necessidades do software final.

        \subsection{Modelagem Organizacional i*}
            % será usado no projeto de IC
            % Ao 
        \subsection{Diagrama de Fluxo de Dados}
            % 

        \subsection{Casos de Uso UML}
            % Gerados pelo JGOOSE ...

        \subsection{Camadas - comunicação ``vertical''}

        \subsection{Arquitetura Modular - Componentes/Subsistemas}
            \label{cap:proposta-sec:maven} %nao deletar - foi comentado no cap 3!
            % tópico
            % reuso de software: Ivonei
            % linhas de produção: Ivonei
            % Apache Maven - ou apenas Maven, mais detalhes na seção seguinte (\ref{cap:proposta-sec:development})
        
            % \subsection{Maven} % modulos, adaptações
                % o que é?
                % como funciona?
                % como usei?
                % que resultados obtive
                % o que esperar?
    % [end]

    \section{Desenvolvimento}
        \label{cap:proposta-sec:development}
        % em módulos maven

        O processo de desenvolvimento foi iterativo e incremental, com apresentações ao Grupo LES (Laboratório de Engenharia de Software).
            Orientado à testes (as vezes)!
            Metodologia Ágil! (TDD)
                Escrever 1o. o teste e ver ele falhar. Se não falhar, já tem algo errado. Nesse caso, falhar é um bom sinal! rs
                Escrever um pedaço de código para passar no teste
                Refatorar / Melhorar o código com segurança, pois o teste já está feito.
            [TODO]
        \subsection{Módulo: JGOOSE}
            Para melhor adaptação e incorporação da ferramenta JGOOSE ao projeto principal do E4J,
                foi necessário adaptar o projeto original do JGOOSE à estrutura de projeto Maven.
        
        \subsection{Módulo: API iStarML}
            % - pq n usar o ccistarml?
                % - JDK5 e código fonte de difícil manutenção (AFD).
            % - justificativas
            % O que caracteriza uma API?
            % O que foi efetivamente desenvolvido?!
            % Annotations e JAXB
            Mapeamento das Tags
                Annotations
                % JAXB
                \emph{Java Architecture for XML Binding} (JAXB) é um padrão Java que ajuda a definir como converter objetos java em XML e vice-versa.
            - restrições da linguagem?

        \subsection{Módulo: Editor}
            JGraphX - exemplo adaptado! % jgraphx detalhado na secao: visao geral
            Grafo
                Nodos
                Arestas
            DIA and Shapes...

            Paletas:
                Atores: ator (e derivados) e links de associação (somente entre atores!)
                Dependências: goal, resource, task, soft-goal, links
                Misc?: notes? annotations? docs?

        
        % \subsection{Módulo: Log}
        %     O módulo Log funcionou como um
        \subsection{\emph{Guidelines} i* Wiki?}
    % [end]
    
    \section{Características da Ferramenta}
        % Alguns recursos presentes na ferramenta merecem uma atenção especial.

        % tópico
        \subsection{Internacionalização (i18n)}
            % o que é?
            % pq é importante?
            % como foi feito?
            % outros pontos importantes?
            % usa padrões de projeto?
        \subsection{UndoManager}
        % \subsection{TODO}
    % [end]

    \section{Considerações Finais do Capítulo}
        \label{cap:proposta-sec:consideracoes}

        Após a compilação completa da ferramenta (junto com suas dependências),
            foi gerado um arquivo executável em JAVA (extensão .jar) com apenas
            ?TODO? megabytes.
            Um número significativamente pequeno, quando comparado aos valores do \emph{i* wiki}, que variam entre ?TODO? e ?TODO?.

        No próximo capítulo será apresentada a ferramenta desenvolvida através de exemplos de uso ...
        
        \subsection{Restrições e Problemas}
            % Avaliação crítica sobre as limitações da Ferramenta!
            % (Fazer isso antes que alguém da banca o faça!)


        % [end]

        \subsection{Trabalhos Futuros}
            \begin{itemize}
                \item 
                \item Desenvolver outras rotinas de importação e exportação de arquivos no módulo \emph{API iStarML}.
            \end{itemize}
        % [end]    
    % [end]

% bibliography, just for auto-link in sublime text
% before compile main.tex, comment line below
\bibliography{../referencias/unioeste,../referencias/tecnologias,../referencias/istar,../referencias/commons,../referencias/books}

% Não pode ser usado o termo 'estudo de caso' para este capítulo!
% "Exemplos de Uso" é igual ao capítulo 5 da Dissertação da Bárbara. (IStarTool)
\chapter{Exemplos de Uso}
    \label{cap:estudo-de-caso}
        % retoma a "chamada" do Cap 1 ?!
        % intro
            % Neste capítulo, ...
        % Qual o tipo de Estudo de Caso? (ver recomendações do Victor) e explicitar no cap 1 !!!
        % Como foi aplicado esse tipo de EC

    \section{Pré-requisitos e Instalação do Sistema}
        % Como instalar?
        % O que é preciso para instalar?

    \section{Conhecendo a Ferramenta}
        % Manual, Telas Numerada, Passo a Passo, etc.

    \section{Usando a Ferramenta}
        % Intro
            % descrever um exemplo! (o próprio TCC?)
        Para fins de exemplo de uso da ferramenta,
            será desenvolvido o mesmo modelo organizacional feito no capítulo anterior - agora com a E4J.

        \subsection{Modelos SD e SR}
            % Desenvolver SD e SR - igual ao cap anterior

        \subsection{Exportando para iStarML}
            % Exportar iStarML e mostrar o resultado arquivo XML

        \subsection{Gerando Casos de Uso com a JGOOSE}
            % Do-it! Saiu igual ao capítulo anterior?

    \section{Resultados e Análises}
        % saiu conforme o esperado?
        % a ferramenta se comportou bem? hehe

    \section{Considerações Finais do Capítulo}

%\chapter{Resultados e An\'{a}lises}
...

% Conclusão
    \chapter{Considerações Finais}
        \label{cap:conclusao}
        % intro
            % Neste capítulo, ...
        % retoma a "chamada" do Cap 1 ?!
    % atingiu os objetivos? ->
        % sim, em tudo
        % sim, parcialmente. mostrar os motivos
        % não. pq? mostrar os motivos
    \section{Contribuições}
        % publicou ou vai publicar?
        Deste trabalho,
            originou a API iStarML
            o JGOOSE em módulo Maven %(benefícios do módulo maven?)

    \section{Trabalhos Futuros}
        % intro
        % \subsection{}
            % intro

        - iStarML:
            I/O de parte do XML (elementos, nodos DOM) e não do arquivo inteiro.




% Inclusão dos apêndices
\oneandhalfspacing
\appendix
%============================================================

\chapter{Como Fazer as Refer�ncias Bibliogr�ficas no Latex}
\label{ApendiceRef}


Neste ap�ndice, mostraremos como as refer�ncias bibliogr�ficas devem ser feitas em Latex. Basta decidir a qual tipo sua refer�ncia pertence (artigo em anais, livros, teses, disserta��es, ...) e preencher os campos adequados para cada tipo. Os campos que est�o com a identifica��o em letras mai�sculas s�o obrigat�rios. Os demais, se voc� tiver as informa��es, melhor; se n�o tiver, basta retir�-lo do tipo.

Para cada refer�ncia, preencha um. Abaixo, um exemplo de cada refer�ncia.

\begin{description}
	\item[Em artigos em revistas \cite{ArtigoRevista}]:
		\small
		\begin{verbatim}
			@ARTICLE{ComoVoceIraReferenciarNoTextoEstaReferencia,
  		  AUTHOR = {Nome Completo do autor1 and Nome Completo do autor2},
  		  TITLE = {T�tulo do Artigo},
  		  JOURNAL = {Nome da revista em que o artigo foi publicado},
  		  YEAR = {Ano},
  		  volume = {Volume},
  		  number = {N�mero},
  		  pages = {pp.inicial-pp.final},
  		  month = {M�s},
  		  address = {Local},
			}
		\end{verbatim}

	\item[Em livros \cite{Livros}]:
		\small
		\begin{verbatim}
			@BOOK{ComoVoceIraReferenciarNoTextoEstaReferencia,
  			AUTHOR = {Nome Completo do autor1 and Nome Completo do autor2},
  			editor = {Quem organizou o livro},
  			TITLE = {T�tulo do livro},
  			PUBLISHER = {Editora},
  			YEAR = {Ano},
  			volume = {Volume},
  			number = {N�mero},
  			series = {S�rie a qual o livro pertence},
  			address = {Local},
  			edition = {N�mero da edi��o},
  			month = {m�s},
			}
		\end{verbatim}

	\item[Cap�tulos em Livros \cite{CapituloLivro}]:
		\small
		\begin{verbatim}
			@INBOOK{ComoVoceIraReferenciarNoTextoEstaReferencia,
  			AUTHOR = {Nome Completo do autor1 and Nome Completo do autor2},
  			editor = {Quem organizou o livro},
  			TITLE = {T�tulo do Livro},
  			CHAPTER = {T�tulo do Cap�tulo},
  			pages = {pp.inicial-pp.final},
  			PUBLISHER = {Editora},
  			YEAR = {Ano},
  			volume = {Volume},	
  			number = {N�mero},
  			series = {S�rie a qual o livro pertence},
  			type = {Tipo. Cap�tulo ou Artigo},
  			address = {Local},
  			edition = {N�mero da edi��o},
  			month = {M�s},
			}
		\end{verbatim}
		
	\item[Artigo em anais \cite{ArtigoAnais}]:
		\small
		\begin{verbatim}
			@INPROCEEDINGS{ComoVoceIraReferenciarNoTextoEstaReferencia,
  			AUTHOR = {Nome Completo do autor1 and Nome Completo do autor2},
  			TITLE = {T�tulo do Artigo publicado},
  			BOOKTITLE = {T�tulo do livro de anais},
  			YEAR = {Ano},
  			editor = {Quem organizou os anais},
  			volume = {Volume},
  			number = {N�mero},
  			series = {S�rie a qual os anais pertencem},
  			pages = {pp.inicial-pp.final},
  			address = {Local},	
  			month = {M�s},
  			organization = {Quem organizou o evento},
  			publisher = {Editora},
			}
		\end{verbatim}

	\item[Manual de SW, procedimentos, etc \cite{Manual}]:
		\small
		\begin{verbatim}
			@MANUAL{ComoVoceIraReferenciarNoTextoEstaReferencia,
  			TITLE = {T�tulo do manual},
  			author = {Nome Completo do autor1 and Nome Completo do autor2},
  			organization = {Empresa e/ou outro},
  			address = {Local},
  			edition = {N�mero da edi��o},
  			month = {M�s},
  			year = {Ano},
			}
		\end{verbatim}

	\item[Relat�rio t�cnico \cite{RelatorioTecnico}]:
		\small
		\begin{verbatim}
			@TECHREPORT{ComoVoceIraReferenciarNoTextoEstaReferencia,
  			AUTHOR = {Nome Completo do autor1 and Nome Completo do autor2},
  			TITLE =  {T�tulo do Relat�rio t�cnico},
  			INSTITUTION =  {Institu��o (empresa, universidade, faculdade, ...)
  			onde o relat�rio foi feito},
  			YEAR = {Ano},
  			type = {Semestral/anual/final},
  			number = {N�mero},
  			address = {Local},
  			month = {M�s},
			}
		\end{verbatim}

	\item[Tese de Doutorado \cite{Tese}]:
		\small
		\begin{verbatim}
			@PHDTHESIS{ComoVoceIraReferenciarNoTextoEstaReferencia,
  			AUTHOR = {Nome Completo do autor1 and Nome Completo do autor2},
  			TITLE = {T�tulo da Tese},
  			SCHOOL = {Universidade},
  			YEAR = {Ano},
  			type = {Tipo. Tese},
  			address = {Local},
  			month = {M�s},
			}
		\end{verbatim}

	\item[Disserta��o de mestrado ou monografia de fim de curso \cite{DissertacaoMonografia}]:
		\small
		\begin{verbatim}
			@MASTERSTHESIS{ComoVoceIraReferenciarNoTextoEstaReferencia,
  			AUTHOR = {Nome Completo do autor1 and Nome Completo do autor2},
  			TITLE = {T�tulo da Disserta��o/Monografia},
  			SCHOOL = {Universidade/Faculdade},
  			YEAR = {Ano},
  			type = {Tipo. Disserta��o ou Monografia},
  			address = {Local},
  			month = {M�s},
			}
		\end{verbatim}

	\item[Refer�ncias que n�o se encaixam em nada \cite{Miscelania}]:
		\small
		\begin{verbatim}
			@MISC{ComoVoceIraReferenciarNoTextoEstaReferencia,
  			author = {Nome Completo do autor1 and Nome Completo do autor2},
  			title = {T�tulo da refer�ncia},
  			howpublished = {Onde foi publicado. Se Internet, colocar: Consultado
  			na Internet em: dd/mm/aaaa},
  			year = {Ano},
  			month = {M�s},
  			note = {alguma nota},
			}
		\end{verbatim}

	\item[Alguma refer�ncia que ainda n�o foi publicada \cite{NaoPublicado}]: Pode ser tamb�m algo que n�o ser� publicado, mas que foi apresentado publicamente, por exemplo, um semin�rio/ trabalho de disciplina.
		\small
		\begin{verbatim}
			@UNPUBLISHED{ComoVoceIraReferenciarNoTextoEstaReferencia,
  			AUTHOR = {Nome Completo do autor1 and Nome Completo do autor2},
  			TITLE = {T�tulo da refer�ncia},
  			NOTE = {alguma nota - por exemplo, onde e quando
  			ser� publicado},
  			year = {Ano},
  			month = {M�s},
			}
		\end{verbatim}
\end{description}



%\begin{attachments}
%\chapter{Guia i*}
\label{anexo:guia}
intro...
origem e justificativa..
sobre a estrutura do guia...
...
o guia 
%\end{attachments}

% incluindo o glossário
% \chapter*{Gloss\'{a}rio}
\addcontentsline{toc}{chapter}{Gloss\'{a}rio}

\begin{tabular}{ll}
Stakeholders &  S\~{a}o todos os envolvidos, pessoas ou organiza\c{c}\~{o}es, que ser\~{a}o afetadas pelo\\
             &  sistema e que possuem influ\^{e}ncia, direta ou indireta, sobre os requisitos
\end{tabular}

% \pagebreak

% Referências Bibliográficas
% \bibliographystyle{abnt-alf}
\bibliographystyle{abnt-num} % use 'abnt-num.bst'
\pagebreak
\addcontentsline{toc}{chapter}{Referências Bibliográficas}

% \citeoption{abnt-etal-cite=2} % only abnt-alf
\bibliography{referencias/abnt-options,referencias/_example,referencias/unioeste,referencias/commons,referencias/books,referencias/tecnologias,referencias/istar}

\end {document}	%% Fim do Documento
