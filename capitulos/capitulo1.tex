% Capítulo 1
\chapter{Introdução}
    \label{cap:introducao}
        % intro
            Este primeiro capítulo tem como objetivo a apresentação geral do trabalho.
            É realizada a contextualização e delimitação da pesquisa ao escopo da Engenharia de Software,
            bem como
            são destacados os principais objetivos da proposta.
        % topicos
            % contexto <#ID_883627655>
                Apresenta-se inicialmente,
                na seção \ref{cap:introducao:sec:contexto},
                o contexto sobre as ferramentas de modelagem organizacional e suas contribuições na área da Engenharia de Requisitos, destacando a influência dessas ferramentas no desenvolvimento de produtos de qualidade.
            % motivacao <#ID_828369918>
                Na seção \ref{cap:introducao:sec:motivacao}, são apresentadas as principais influências e motivações para a realização do trabalho.
            % proposta <#ID_1542036198>
                Em seguida,
                na seção \ref{cap:introducao:sec:proposta},
                é apresentada a proposta sob uma visão geral e os objetivos norteadores da pesquisa.
            % contribuicoes <#ID_1701692394>
                Na seção \ref{cap:introducao:sec:contribuicoes},
                 descreve-se as contribuições esperadas após a finalização deste trabalho.
            % organizacao <#ID_1115413878>
                Por fim,
                 na seção \ref{cap:introducao:sec:organizacao},
                é apresentada a estrutura geral e a organização do restante desta monografia.
    \section{Contexto}
        \label{cap:introducao:sec:contexto}
        % Contexto - Ferramentas
            Muitas são as opções de ferramentas e técnincas que visam auxiliar engenheiros de requisitos no processo de construção de modelos organizacionais i* \cite{site2013iwiki} \cite{grau2006comparative}.
            Podendo ser classificadas como ferramentas CASE (\emph{Computer-Aided Software Engineer})
                \footnote{Ferramentas CASE, é toda e qualquer ferramenta baseada em computador que auxilie nas atividades de desenvolvimento de software.}
             \cite{case1985computer}, essas ferramentas têm como objetivo o aumento da produtividade e a melhoria da qualidade final dos softwares,
                 através da automatização e gerenciamento de várias fases da Engenharia de Software.
        
        % Contexto - E.R.
            A área de Engenharia de Requisitos (ER), subárea da Engenharia de Software, é responsável por diversas atividades que abrangem os processos de análise, elicitação, especificação, avaliação, ajuste, documentação e evolução dos requisitos de um sistema computacional.
            É uma das áreas mais críticas para o sucesso e qualidade de um projeto de software \cite{sommerville1998requirements}.
            Pesquisas pertinentes à ER, tanto em âmbito acadêmico quanto industrial,
            apontam a falta de um entendimento adequado da organização por parte dos responsáveis pela elaboração do documento de requisitos
            como sendo uma das principais falhas no processo de especificação dos requisitos
                \cite{van2000requirements}.

        % Contexto - solução: modelagem organizacional
            % intro
                Para tentar diminuir os problemas relacionados as fases iniciais do projeto,
                pesquisas recentes mostram que a comunidade tem buscado estabelecer e utilizar padrões de técnicas, métodos e ferramentas para tratar especificamente
                da fase inicial de desenvolvimento de software
                    % ref
                        % A Literature Survey on International Standards for Systems Requirements Engineering
                        \cite{schneider2013literature}.
                Pensando nisso,
                têm-se investido esforços no processo de modelagem organizacional.
            % conceito
                Este tipo de modelagem
                visa prover recursos que permitam modelar
                as intenções, relacionamentos e motivações
                entre membros de uma organização
                % refs
                    % A Conceptual Basis for Organizational Modelling
                    % mason1997conceptual <workspace:/../../../../E:/unioeste/BCC/TCC-2013/referencias/mason1997conceptual.pdf>
                    \cite{mason1997conceptual}.
            % modelagem i*
                Dentre as técnicas de modelagem organizacional,
                destaca-se a i*,
                proposto por
                    \cite{yu1993modeling},
                uma técnica que utiliza a orientação a agentes
                    \cite{yu2001agent}
                %
                    com enfoque
                    tanto nos desejos e intenções desses agentes, quanto suas dependências
                    %refs
                        \cite{site2013iwiki}
                        \cite{yu1997towards}.
                Mais detalhes sobre esta técnica e suas variações serão discutidos no capítulo
                \ref{cap:framework}.
        % Contexto - Mapeamento Modelagem / UseCase UML

            Pensando em auxiliar no processo de desenvolvimento de software,
            alguns trabalhos foram propostos com o intuito de realizar o mapeamento
            de modelos do \emph{framework} i*
            para diagramas da UML (do inglês \emph{Unified   Modeling   Language}).
            Dentre esses trabalhos, destaca-se o trabalho de Santander \cite{santander2002integrando},
            que propõe a derivação em casos de uso UML a partir de modelos do \emph{framework} i*.
        % Contexto - JGOOSE
            % intro
                Para apoiar esse processo de derivação,
                foi desenvolvida a ferramenta JGOOSE (Java Goal into Object Oriented Standard Extension) \cite{vicente2006},
                uma ferramenta que mapeia de forma automática os diagramas i* para casos de uso UML.
                Essa ferramenta tem como base as diretrizes e passos propostos por Santander \cite{santander2002integrando},


                Inicialmente apresentada como GOOSE (\emph{Goal into Object Oriented Standard Extension}) em
                    \cite{pedroza2004ferramentas}
                    e
                    \cite{brischke2005desenvolvimento},
                em seguida, melhorada e apresentada como JGOOSE por
                    \cite{vicente2009},
                passou também por melhorias com
                    \cite{brischke2012melhorando}
                e, atualmente,
                está sendo aprimorada por
                    \cite{peliser2013aprimorando}.

                Ou seja, dado como entrada os modelos i*, no formato de arquivo TELOS
                    \cite{mylopoulos1990telos}
                    \cite{koubarakis1989telos},
                a ferramenta consegue gerar conforme o template proposto em
                    %ref
                    \cite{cockburn2001writing}
                os casos de uso UML com um bom nível de detalhamento.
            % problema 1 - dependência

                Porém, a ferramenta JGOOSE ainda não possui funcionalidades para a produção dos arquivos de entrada da ferramenta.
                Ainda existe a dependência da ferramenta OME ou, mais especificamente, ao formato de arquivo TELOS.
            % solução 1 - editor integrado
                Nesse contexto, percebe-se a necessidade de se desenvolver um editor de modelos i* integrado à ferramenta JGOOSE,
            % solução 2 - iStarML
                bem como implementar o suporte à especificação do formato de arquivo iStarML
                    % referenciar a proposta inicial do istarml
                    \cite{cares2007istarml},
                uma formato em XML
                para representação de modelos i*
                com o propósito de servir como um intercâmbio entre os outros meta-modelos existentes
                    \cite{colomer2011model}.
    \section{Motivação}
        \label{cap:introducao:sec:motivacao}
        % backlink - resumo <#ID_1432075869>
        % O que me motivou?
            % Motivo, Causa, Razão ou Circunstância
            % É uma área em crescimento e destaque. Muito esforço já se investiu em ferramentas CASE.
            % PQ é valido o esforço para aprimorar a ferramenta?
        % Motivacional 0 - Resolver algum dos Problemas mencionados no contexto
            % JGOOSE não possui um editor i*
            % Dependência de outras ferramentas p/ ler Telos
            % iStarML
        % Motivacional 1 - Ajudar a comunidade
            A área de Engenharia de Requisitos está em crescimento e destaque
            por impactar de forma tão significativa nos resultados finais de um projeto de software.
            Desta forma,
            é valido o investimento de esforços para a melhoria de métodos, técnicas ou ferramentas que auxiliem os profissionais da área a aprimorar seu trabalho de forma eficiente.
            % framework i*
                % padrão internacional
                    O i* é a base da GRL (\emph{Goal-oriented Requirements Language} ou Linguagem de Requisitos Orientada a Objetivos),
                    que junto à UCM
                        \footnote{UCM - \emph{Use Case Maps} , em português: ``Mapas de Caso de Uso''. Uma técnica de engenharia de software baseada em cenários para descrever relacionamentos entre um ou mais casos de uso.}
                    constituíram a URN
                        \footnote{URN - \emph{User Requirements Notation}, em portugês: ``Notação Requisitos de Usuário''. Notação destinada a elicitação, análise, especificação e validação de requisitos.},
                    que passou a ser adotada como um padrão internacional,
                    em novembro de 2008,
                    pela ITU (\emph{International Telecommunication Union})
                    % Z.151 : User Requirements Notation (URN) - Language definition
                    % refs
                        \cite{amyot2003introduction}
                        \cite{itu2003urn}. % http://www.itu.int/rec/T-REC-Z.151/en
                % motivação
                    Além de se tratar de um padrão internacional,
                    é a técnica de modelagem já justificada pela JGOOSE
                    e seus usuários já estão familiarizados com os conceitos do \emph{framework}.
        % Motivacional 2 - Ajudar os alunos

            Além disso,
            a ferramenta de que trata este trabalho
            é frequentemente usada por acadêmicos do curso de Ciência da Computação da Universidade Estadual do Oeste do Paraná - Campus Cascavel.
            Todos os anos, alunos se debatem com problemas apresentados por outros softwares de modelagem i*.
            Os questionamentos mais comuns estão relacionados à usabilidade e integridade das ferramentas.
            Isso acarreta em oportunidades para novas soluções computacionais se apresentarem.
        % Motivacional 3 - Estudar sobre a área

            Outro fator, não menos importante,
            é o gosto pessoal pela área de projeto e desenvolvimento de software.
            Isto ajudou na tomada de decisão quanto ao foco da pesquisa, bem como resultou na implementação de uma API para o iStarML (Veja o apêndice \ref{apendice:istarml}).
    \section{Proposta}
        \label{cap:introducao:sec:proposta}
        % backlink - resumo <#ID_1921781916>
        % intro JGOOSE
            A ferramenta JGOOSE,
            no escopo do seu propósito,
            já atende as principais necessidades do engenheiro de requisitos.
            Porém, ainda existe a dependência da ferramenta mencionada (OME)
            para elaborar os modelos organizacionais e exportá-los em arquivo TELOS.
        % estender JGOOSE (Obj. Geral)

            Dessa forma, o objetivo geral deste trabalho consiste em
            aumentar os recursos e funcionalidades da ferramenta JGOOSE através do
            desenvolvimento de uma nova interface para edição de modelos do framework i* integrada à JGOOSE.
            Ou seja, prover aos usuários da ferramenta JGOOSE uma interface gráfica rica em recursos que facilitem o trabalho de modelagem organizacional, visando diminuir a necessidade de usar outros softwares para esse fim.
        %Objetivos Específicos
            
             Como objetivos específicos, têm-se:
            \begin{itemize}
            % estudar literatura da área
                \item Estudar sobre as ferramentas de modelagem organizacional i* disponíveis à comunidade.
            % estudar sobre o framework i* e suas variações
                \item Estudar sobre o framework i*, bem como suas variações.
            % estudar a iStarML
                \item Estudar o formato de arquivo iStarML e incorporá-lo à ferramenta como o formato de arquivo padrão.
            % estudar a JGOOSE e sua evolução/histórico
                \item Realizar um estudo sobre a evolução histórica da ferramenta JGOOSE e analisar sua arquitetura na versão 2013.
            % criar e apresentar exemplos de uso da ferramenta
                \item Criar e apresentar exemplos de uso da ferramenta proposta, a fim de verificar as principais funcionalidades da ferramenta na versão final.
            \end{itemize}
    \section{Contribuições Esperadas}
        \label{cap:introducao:sec:contribuicoes}
        % backlink - resumo <#ID_491307769>
        Após a finalização deste trabalho, deseja-se
        % Ferramenta
            % elimitar ou diminuir a dependência de outras ferramentas
            uma maior independência para a JGOOSE e, consequentemente, seus usuários,
            % Maior facilidade no uso da ferramenta JGOOSE NOOP
            além de aumentar o destaque na comunidade e promover a adoção da ferramenta para fins de modelagem i*.
        % Eng de Requisitos
            % melhor rastreamento dos requisitos ?
        % Reconhecimento
            % Comunidade i* - i* wiki
            Por fim, espera-se uma contribuição significativa diante das ferramentas da comunidade i*,
            trazendo um reconhecimento para o grupo de pesquisa do LES e todos os envolvidos no desenvolvimento e progresso da JGOOSE.
            Além disso, seria uma nova ferramenta no quadro comparativo do i* Wiki \cite{site2013iwiki}.
    \section{Estrutura do Trabalho}
        \label{cap:introducao:sec:organizacao}
        Basicamente,
        o restante deste trabalho encontra-se organizado da seguinte maneira:
            Nos capítulos \ref{cap:framework} e \ref{cap:jgoose} são apresentados alguns fundamentos teóricos necessários para uma melhor compreensão da área de estudo de que trata este trabalho.  No capítulo \ref{cap:framework}, os conceitos básicos e as características do \emph{Framework} i*, bem como suas variações, são apresentados.  Já no capítulo \ref{cap:jgoose}, após um estudo sobre a evolução histórica do software JGOOSE, uma análise e discussão detalhada de sua arquitetura, na versão 2013, é realizada.
            No capítulo \ref{cap:proposta}, a proposta é detalhada através de uma visão geral do projeto e arquitetura da nova interface. Também é apresentada uma discussão sobre os principais recursos disponíveis aos usuários.
            Em seguida, no capítulo \ref{cap:estudo-de-caso} é apresentado um estudo de caso, mostrando e avaliando a aplicação da ferramenta em um domínio específico.
            Finalmente, o capítulo \ref{cap:conclusao} reúne as análises e considerações finais sobre os resultados, bem como relata sobre os possíveis trabalhos futuros.

        Outra composição importante, do ponto de vista técnico-computacional, é o apêndice \ref{apendice:istarml}: uma documentação sobre a API desenvolvida durante a fase de implementação.
        