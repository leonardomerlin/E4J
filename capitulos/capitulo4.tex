% 
% Escolhas e JUSTIFICATIVAS!?
% 
\chapter{E4J - Editor i* para JGOOSE}
    \label{cap:proposta}
    % intro
        Este capítulo apresenta a solução computacional \ferramenta{}, um editor de modelos organizacionais i* integrado à ferramenta JGOOSE.

        Para a apresentação desta proposta, este capítulo foi dividido nas seguintes seções:
            \begin{itemize}
                \item ... % TODO
            \end{itemize}
    % [end]

    \section{Visão Geral}
        \label{cap:proposta-sec:overview}
        % tópico - intro

        % nesse sentido,
        %   adotou-se 
        A E4J (pronunciado ``i-fór-jey''), representa a abreviação de ``Editor para JGOOSE'', mais especificamente: \textbf{E}ditor i* \textbf{para} \textbf{J}GOOSE.


        Para se usar com efetividade a \ferramenta{},
            é necessário conhecer os conceitos originais da modelagem organizacional i* propostos por Yu \cite{yu1995modelling}, vistos no capítulo \ref{cap:framework} deste trabalho.
            Também é necessário conhecer o funcionamento da ferramenta JGOOSE, visto no capítulo \ref{cap:jgoose}, para entender o mapeamento dos casos de uso UML.

        % \subsection{Considerações Técnicas}
        %     A E4J foi desenvolvida na linguagem de programação JAVA \cite{java}.
        %         % A JGOOSE ja é java,
        %         % ...
        %         % vantagens e beneficios de se manter a Linguagem JAVA

        \subsection{JGraphX}
            A JGraphX é uma Biblioteca para Java Swing (\emph{Java Swing Library}) da \emph{mxGraph} \cite{mxgraph}.
                Consiste em um conjunto de estruturas e funcionalidades que facilitam a produção de Aplicações Java Swing \cite{javaswing}.
                Com essa biblioteca é possível criar aplicações interativas voltadas principalmente para manipulação de diagramas.

            O núcleo dessa biblioteca é fundamentada na Teoria dos Grafos \cite{teoria dos grafos}.
                Um grafo consiste em um conjunto de vertices, chamados também de nodos ou nós, e arestas.
                As arestas são conexões entre os nós.
                Na estrutura da JGraphX, existe também o conceito de célula, que representa um elemento do grafo: arestas, vértices ou um grupo destes.

                Essa biblioteca foi escolhida pelos seguintes fatores:
                    \begin{itemize}
                        \item é uma biblioteca Java focada em aplicações para manipulação de diagramas sobre a estrutura de grafos. Como os modelos organizacionais i* podem ser construídas sobre estruturas de grafos, este ítem é relativamente significante para a escolha da biblioteca.
                        \item boa representatividade na comunidade. Essa biblioteca é citada em diversos fóruns da área e, recentemente, foi realizado um estudo comparativo entre a biblioteca JGraphX e a JUNG, ambas soluções para visualização e manipulação de grafos \cite{souza2013analise}. Nesse estudo, segundo os critérios avaliados, a JGraphX apresentou melhores caraterísticas que a JUNG.
                        \item está sob a licensa BSD \cite{bsd}.
                        \item é disponibilizada em repositório Maven com boa documentação (JavaDoc) e código-fonte.
                    \end{itemize}
    % [end]

    \section{Projeto e Arquitetura}
        \label{cap:proposta-sec:design}
        % visões: SD, SR; DFD; Camadas; Componentes;
            Antes de começar a desenvolver efetivamente a \ferramenta{},
                foram desenvolvidos alguns modelos que auxiliaram o entendimento das necessidades do software final.

        \subsection{Modelagem Organizacional i*}
            % será usado no projeto de IC
            % Ao 
        \subsection{Diagrama de Fluxo de Dados}
            % 

        \subsection{Casos de Uso UML}
            % Gerados pelo JGOOSE ...

        \subsection{Camadas - comunicação ``vertical''}

        \subsection{Arquitetura Modular - Componentes/Subsistemas}
            \label{cap:proposta-sec:maven} %nao deletar - foi comentado no cap 3!
            % tópico
            % reuso de software: Ivonei
            % linhas de produção: Ivonei
            % Apache Maven - ou apenas Maven, mais detalhes na seção seguinte (\ref{cap:proposta-sec:development})
        
            % \subsection{Maven} % modulos, adaptações
                % o que é?
                % como funciona?
                % como usei?
                % que resultados obtive
                % o que esperar?
    % [end]

    \section{Desenvolvimento}
        \label{cap:proposta-sec:development}
        % em módulos maven

        O processo de desenvolvimento foi iterativo e incremental, com apresentações ao Grupo LES (Laboratório de Engenharia de Software).
            Orientado à testes (as vezes)!
            Metodologia Ágil! (TDD)
                Escrever 1o. o teste e ver ele falhar. Se não falhar, já tem algo errado. Nesse caso, falhar é um bom sinal! rs
                Escrever um pedaço de código para passar no teste
                Refatorar / Melhorar o código com segurança, pois o teste já está feito.
            [TODO]
        \subsection{Módulo: JGOOSE}
            Para melhor adaptação e incorporação da ferramenta JGOOSE ao projeto principal do E4J,
                foi necessário adaptar o projeto original do JGOOSE à estrutura de projeto Maven.
        
        \subsection{Módulo: API iStarML}
            % - pq n usar o ccistarml?
                % - JDK5 e código fonte de difícil manutenção (AFD).
            % - justificativas
            % O que caracteriza uma API?
            % O que foi efetivamente desenvolvido?!
            % Annotations e JAXB
            Mapeamento das Tags
                Annotations
                % JAXB
                \emph{Java Architecture for XML Binding} (JAXB) é um padrão Java que ajuda a definir como converter objetos java em XML e vice-versa.
            - restrições da linguagem?

        \subsection{Módulo: Editor}
            JGraphX - exemplo adaptado! % jgraphx detalhado na secao: visao geral
            Grafo
                Nodos
                Arestas
            DIA and Shapes...

            Paletas:
                Atores: ator (e derivados) e links de associação (somente entre atores!)
                Dependências: goal, resource, task, soft-goal, links
                Misc?: notes? annotations? docs?

        
        % \subsection{Módulo: Log}
        %     O módulo Log funcionou como um
        \subsection{\emph{Guidelines} i* Wiki?}
    % [end]
    
    \section{Características da Ferramenta}
        % Alguns recursos presentes na ferramenta merecem uma atenção especial.

        % tópico
        \subsection{Internacionalização (i18n)}
            % o que é?
            % pq é importante?
            % como foi feito?
            % outros pontos importantes?
            % usa padrões de projeto?
        \subsection{UndoManager}
        % \subsection{TODO}
    % [end]

    \section{Considerações Finais do Capítulo}
        \label{cap:proposta-sec:consideracoes}

        Após a compilação completa da ferramenta (junto com suas dependências),
            foi gerado um arquivo executável em JAVA (extensão .jar) com apenas
            ?TODO? megabytes.
            Um número significativamente pequeno, quando comparado aos valores do \emph{i* wiki}, que variam entre ?TODO? e ?TODO?.

        No próximo capítulo será apresentada a ferramenta desenvolvida através de exemplos de uso ...
        
        \subsection{Restrições e Problemas}
            % Avaliação crítica sobre as limitações da Ferramenta!
            % (Fazer isso antes que alguém da banca o faça!)


        % [end]

        \subsection{Trabalhos Futuros}
            \begin{itemize}
                \item 
                \item Desenvolver outras rotinas de importação e exportação de arquivos no módulo \emph{API iStarML}.
            \end{itemize}
        % [end]    
    % [end]

% bibliography, just for auto-link in sublime text
% before compile main.tex, comment line below
\bibliography{../referencias/unioeste,../referencias/tecnologias,../referencias/istar,../referencias/commons,../referencias/books}
