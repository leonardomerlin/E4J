\chapter*{Resumo}
\addcontentsline{toc}{chapter}{Resumo}
\noindent

Em discussões pertinentes à engenharia de requisitos, tanto em âmbito
acadêmico quanto industrial, ressalta-se que um dos principais desafios da área
consiste na integração de modelos organizacionais às demais etapas do processo
de engenharia de requisitos.

Alguns trabalhos relacionados já apresentaram técnicas para a realização dessa integração. Nesse sentido, em 2006 foi
desenvolvida uma solução computacional denominada JGOOSE (do inglês Java Goal Into Object Oriented Standard Extension),
uma ferramenta de auxílio a engenharia de requisitos que proporciona a automatização do 
mapeamento de modelos e diagramas do framework i* para casos de uso UML.

Apesar das atualizações e melhorias realizada por outros pesquisadores e membros
do grupo do Laboratório de Engenharia de Software (LES) da Universidade Estadual do Oeste do Paraná (UNIOESTE - Campus Cascavel/PR),
a ferramenta ainda possui uma dependência crítica em relação à elaboração dos dados de entrada.
Nas condições atuais é necessário instalar alguma ferramenta específica para produzir o arquivo no formato TELOS.

Desta forma,
	este trabalho consiste no projeto e desenvolvimento de um editor gráfico de modelos organizacionais i* integrado à ferramenta JGOOSE, visando melhorar suas funcionalidades e diminuir a necessidade de outros softwares para este fim.
	Uma característica importante desse novo editor é o suporte à especificação iStarML - um formato de arquivo baseado em XML para representação e intercâmbio de modelos i*.


\vspace{1cm}

\noindent
\textbf{Palavras-chave: } E4J, JGOOSE, iStarML, Modelagem Organizacional, Engenharia de Requisitos.