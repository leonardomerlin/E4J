\chapter*{Resumo}
\addcontentsline{toc}{chapter}{Resumo}
\noindent

Em discuss\~{o}es  pertinentes  \`{a}  engenharia de requisitos, tanto em \^{a}mbito
acad\^{e}mico quanto industrial, ressalta-se que um dos principais desafios da \'{a}rea
consiste na integra\c{c}\~{a}o de modelos organizacionais \`{a}s demais  etapas do processo
de engenharia de requisitos.

Alguns trabalhos relacionados j\'{a} apresentaram t\'{e}cnicas para a realiza\c{c}\~{a}o dessa integra\c{c}\~{a}o. Nesse sentido, em 2006 foi
desenvolvida uma solu\c{c}\~{a}o computacional denominada JGOOSE (do ingl\^{e}s  Java Goal Into Object  Oriented Standard Extension),
uma ferramenta de aux\'{\i}lio a engenharia de requisitos que proporciona a automatiza\c{c}\~{a}o do 
mapeamento de modelos e diagramas do  framework  i* para casos de uso UML.

Apesar das atualiza\c{c}\~{o}es e melhorias realizada por outros pesquisadores e membros
do grupo do Laborat\'{o}rio de Engenharia de Software (LES) da Universidade Estadual do Oeste do Paran\'{a} (UNIOESTE - Campus Cascavel/PR),
a ferramenta ainda possui uma depend\^{e}ncia cr\'{\i}tica quanto a elabora\c{c}\~{a}o dos dados de entrada.
Ou seja, nas condi\c{c}\~{o}es atuais \'{e} necess\'{a}rio instalar alguma ferramenta espec\'{\i}fica para produzir o arquivo no formato TELOS.

Desta forma, este trabalho consiste no projeto e desenvolvimento de um editor gr\'{a}fico de modelos organizacionais i* integrado  \`{a} ferramenta JGOOSE, visando melhorar
suas funcionalidades e diminuir a necessidade de outros softwares para este fim.
Uma caracter\'{\i}stica importante desse novo editor \'{e} o suporte \`{a} especifica\c{c}\~{a}o
iStarML - um formato de arquivo baseado em XML para representa\c{c}\~{a}o de modelos i*.


\vspace{1cm}

\noindent
\textbf{Palavras-chave: } JGOOSE, iStarML, i* Framework, Modelagem Organizacional, Engenharia de Requisitos.